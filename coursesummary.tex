\documentclass[11pt,a4paper]{article}

% --- FONT + LANGUAGE ---
\usepackage{fontspec} % for custom fonts (requires xelatex/lualatex)
\setmainfont{Arial} % <-- requires Kalam installed
\usepackage{graphicx}   % for including images
\usepackage{float}      % optional: for [H] placement
\usepackage{xcolor}
\usepackage{tikz-cd} 
% --- MATH PACKAGES ---
\usepackage{amsmath,amssymb,amsthm}
\usepackage{mathtools}
\usepackage{physics}   % nice shorthand like \dv, \pdv
\usepackage{bm}        % bold math

% --- PAGE & STYLE ---
\usepackage[a4paper,margin=1in]{geometry}
\usepackage{titlesec} % custom section titles
\usepackage{fancyhdr} % headers/footers
\usepackage{xcolor}   % colors

% --- HEADER / FOOTER ---
\pagestyle{fancy}
\fancyhf{}
\lhead{\textbf{Course Summary}}
\rhead{\leftmark}
\cfoot{\thepage}

% --- THEOREM ENVIRONMENTS ---
\newtheorem*{theorem}{Theorem}
\newtheorem*{lemma}{Lemma}
\newtheorem*{definition}{Definition}
\newtheorem*{example}{Example}
\newtheorem*{exercise}{Exercise}
\newtheorem*{proposition}{Proposition}
\newtheorem*{corollary}{Corollary}
\newtheorem*{remark}{Remark}


% --- CUSTOM MACROS ---
\newcommand{\lecture}[2]{
	\section*{Lecture #1 -- #2}
	\addcontentsline{toc}{section}{Lecture #1: #2}
}
% --- INTERMEZZO (non-lecture sections) ---
% --- INTERMEZZO SECTIONS (bold, main-level, no lines) ---
\newcommand{\intermezzo}[2]{
	\clearpage
	\section*{Intermezzo #1 -- #2}
	\addcontentsline{toc}{section}{Intermezzo #1: #2}
	\vspace{1em}
}

\newcommand{\solution}[1]{
	\subsection*{Solution}
	#1
}
\newcommand{\problems}[1]{
	\subsection*{Problems}
	#1
}

% --- DOCUMENT ---
\begin{document}
	
	\begin{center}
		{\Huge \textbf{Course Summary}} \\[1em]
		{\Large Advanced Geometry 1 (Algebraic Topology Introduction)} \\[0.5em]
		{\large Andrew Moyer} \\[2em]
	\end{center}
	
	\tableofcontents
	\newpage
	
	% --- SAMPLE CONTENT ---
	\lecture{0}{Preliminary Definitions from Munkres Topology Chapter 2}
	% ============================
	% --- PRELIMINARY DEFINITIONS
	% ============================
	
	\begin{definition}[Topology]
		Let $X$ be a set. A \emph{topology} on $X$ is a collection $\mathcal{T}$ of subsets of $X$ (called \emph{open sets}) satisfying:
		\begin{enumerate}
			\item $\varnothing, X \in \mathcal{T}$,
			\item the union of any collection of sets in $\mathcal{T}$ is in $\mathcal{T}$,
			\item the intersection of any finite collection of sets in $\mathcal{T}$ is in $\mathcal{T}$.
		\end{enumerate}
		The pair $(X,\mathcal{T})$ is called a \emph{topological space}.
	\end{definition}
	
	\begin{definition}[Basis of a Topology]
		A collection $\mathcal{B}$ of subsets of $X$ is a \emph{basis} for a topology on $X$ if:
		\begin{enumerate}
			\item for each $x \in X$, there exists $B \in \mathcal{B}$ with $x \in B$,
			\item if $x \in B_1 \cap B_2$ with $B_1,B_2 \in \mathcal{B}$, then there exists $B_3 \in \mathcal{B}$ such that $x \in B_3 \subseteq B_1 \cap B_2$.
		\end{enumerate}
		The topology \emph{generated} by $\mathcal{B}$ is the collection of all unions of elements of $\mathcal{B}$.
	\end{definition}
	
	\begin{definition}[Subspace Topology]
		Let $(X,\mathcal{T})$ be a topological space and let $Y \subseteq X$. The \emph{subspace topology} on $Y$ is
		\[
		\mathcal{T}_Y = \{\,U \cap Y : U \in \mathcal{T}\,\}.
		\]
	\end{definition}
	
	\begin{definition}[Closed Set]
		A subset $A \subseteq X$ is called \emph{closed} if its complement $X \setminus A$ is open.
	\end{definition}
	
	\begin{definition}[Topology via Closed Sets]
		A collection $\mathcal{C}$ of subsets of $X$ is the collection of \emph{closed sets} of a topology on $X$ if:
		\begin{enumerate}
			\item $\varnothing, X \in \mathcal{C}$,
			\item the intersection of any collection of sets in $\mathcal{C}$ is in $\mathcal{C}$,
			\item the union of any finite collection of sets in $\mathcal{C}$ is in $\mathcal{C}$.
		\end{enumerate}
	\end{definition}
	
	\begin{definition}[Closure and Interior]
		Let $A \subseteq X$.
		\begin{itemize}
			\item The \emph{closure} of $A$, denoted $\overline{A}$, is the intersection of all closed sets containing $A$.
			\item The \emph{interior} of $A$, denoted $A^{\circ}$, is the union of all open sets contained in $A$.
		\end{itemize}
	\end{definition}
	
	\begin{definition}[Hausdorff Space]
		A topological space $(X,\mathcal{T})$ is called \emph{Hausdorff} (or $T_2$) if for every pair of distinct points $x,y \in X$, there exist disjoint open sets $U,V \in \mathcal{T}$ such that $x \in U$ and $y \in V$.
	\end{definition}
	
	\begin{definition}[Continuous Function]
		Let $(X,\mathcal{T}_X)$ and $(Y,\mathcal{T}_Y)$ be topological spaces. A function $f : X \to Y$ is \emph{continuous} if any of the following equivalent conditions hold:
		\begin{enumerate}
			\item for every open set $V \subseteq Y$, the preimage $f^{-1}(V)$ is open in $X$;
			\item for every closed set $C \subseteq Y$, the preimage $f^{-1}(C)$ is closed in $X$;
			\item for every $x \in X$ and every neighborhood $V$ of $f(x)$ in $Y$, there exists a neighborhood $U$ of $x$ in $X$ such that $f(U) \subseteq V$.
		\end{enumerate}
	\end{definition}
	
	\begin{definition}[Homeomorphism]
		A function $f : X \to Y$ between topological spaces is a \emph{homeomorphism} if it is bijective, continuous, and its inverse $f^{-1}$ is also continuous.  
		Two spaces are \emph{homeomorphic} if there exists a homeomorphism between them; they are then considered topologically equivalent.
	\end{definition}
	
	\begin{definition}[Metric]
		A \emph{metric} on a set $X$ is a function $d : X \times X \to [0,\infty)$ satisfying, for all $x,y,z \in X$:
		\begin{enumerate}
			\item $d(x,y) = 0$ iff $x = y$;
			\item $d(x,y) = d(y,x)$ (symmetry);
			\item $d(x,z) \le d(x,y) + d(y,z)$ (triangle inequality).
		\end{enumerate}
		The pair $(X,d)$ is called a \emph{metric space}.
	\end{definition}
	
	\begin{definition}[Metric Topology]
		Given a metric space $(X,d)$, the \emph{metric topology} $\mathcal{T}_d$ is the collection of all subsets $U \subseteq X$ such that for every $x \in U$, there exists $\varepsilon > 0$ with the open ball
		\[
		B_d(x,\varepsilon) = \{\,y \in X : d(x,y) < \varepsilon\,\} \subseteq U.
		\]
	\end{definition}
	
	\begin{definition}[Quotient Map and Quotient Topology]
		Let $X$ and $Y$ be topological spaces and $q : X \to Y$ a surjective map.
		\begin{enumerate}
			\item The map $q$ is a \emph{quotient map} if a subset $U \subseteq Y$ is open in $Y$ iff $q^{-1}(U)$ is open in $X$.
			\item If $q$ is surjective, the \emph{quotient topology} on $Y$ induced by $q$ is defined by
			\[
			\mathcal{T}_Y = \{\,U \subseteq Y : q^{-1}(U) \text{ is open in } X\,\}.
			\]
		\end{enumerate}
	\end{definition}
	
	\lecture{1}{Introduction to Algebraic Topology}
	The big question. Given two topological spaces $X,Y$ (likely manifolds), are they homeomorpic? Does there exists $f:X\to Y$ a homeomorphism? To prove this, one must construct a homeomorphism. But to disprove it, we can use topological invariants and show that $X$ and $Y$ are different. We must be able to construct and calculate them. Given a topological space, we can construct maps to number systems, polynomials, groups, vector spaces and so on. If we can prove that these properties are preserved under homeomorphism, we have a topological invariant
	\newline
	
	In the course we will cover simplicial homolgy for polyhedra, singular homology for all topological spaces and cellular homology for CW complexes. At the end we will study cohomology, which is the geometric dual of homology. To begin simplicial homology, we must start with the building blocks called simplices.
	\begin{definition}
		Given $X=\{a_{0},...,a_{n}\} \subseteq \mathbb{R}^{n}$, $X$ is said to be geometrically independent if the following equations hold true:
		\begin{itemize}
			\item $\sum_{i=0}^{n}t_{i} = 0$
			\item $\sum_{i=0}^{n}t_{i}a_{i} = 0$
		\end{itemize}
		with $t_{i}\in\mathbb{R}$ imply that $t_{1}+t_{2}+...+t_{n} = 0$
	\end{definition}
	The set $X$ is just a bunch of points in $\mathbb{R}^{n}$. If we subtract the set from a fixed point then we get a set of linearly independent vectors. 
	\begin{definition}
Let $\{a_{0},...,a_{n}\}$ be a set and consider
$$
\{x = \sum_{i=0}^{n}t_{i}a_{i}|\sum_{i=0}^{n}t_{1}=1 \text{ and } t_{i}\geq 0\forall i\}
$$
This is called an \textit{n simplex} generated by $a_{0},...,a_{n}$
	\end{definition}
	One can do a little algebra with a triangle and show that this sweeps out the edges and interior.
	\begin{figure}[H]
		\centering
		\includegraphics[width=0.3\textwidth]{figures/sweep.png}
		\label{fig:my_diagram}
	\end{figure}
	In general it will sweep out the polyhedra and interior of with the geometrically independent set as the vertices.
	\begin{definition}
		Let $\sigma$  be an $n$-simplex generated by $a_{0},...,a_{n}$
\begin{itemize}
\item $n$ is the dimension of the simplex
\item The simplex $\{a_{i_{0}},...a_{i_{k}}\}\subseteq\{a_{0},...,a_{k}\}$ is a k dimensional face of $\sigma$
\end{itemize}
	\end{definition}
	Finally, we define a simplicial complex so it glues together in a way that makes sense.
\begin{definition}
A simplicial complex $K\subset\mathbb{R}^{n}$ is a set of simplices in $\mathbb{R}^{n}$ such that
\begin{itemize}
\item $\sigma\in K$, $\sigma^{\prime}$ a face of $K \Rightarrow \sigma^{\prime}\in K$
\item $\sigma,\sigma^{\prime}\in K$ and $\sigma\cap\sigma^{\prime}\neq\emptyset \Rightarrow \sigma\cap\sigma^{\prime}$ is a face common to both $\sigma$ and $\sigma^{\prime}$
\end{itemize}
\end{definition}
A few examples:
	\begin{figure}[H]
	\centering
	\includegraphics[width=0.5\textwidth]{figures/simplicialcomplexex1.png}
\end{figure}
	\begin{figure}[H]
	\centering
	\includegraphics[width=0.5\textwidth]{figures/simplicialcomplexex2.png}
	\label{fig:my_diagram}
\end{figure}
	\lecture{2}{Topologies of Simplicial Complexes, Free Abelian Groups and Homology Group Introduction}
	In previous lecture we defined as simplical complex $K$. Now we define a topology on $K$
	\begin{definition}
		Let $\displaystyle\bigcup_{\sigma \in K} \sigma$
		\begin{itemize}
		\item The topology of each $\sigma$ is induced by the standard topology of $\mathbb{R}^{n}$
		\item $A\subseteq |K|$ is closed iff $A\cap\sigma$ is closed for all $\sigma\in K$
		\end{itemize}
	\end{definition}
	So each individual $\sigma$ just has the topology from little $\epsilon$-balls in $\mathbb{R}^{n}$ restricted to the subspace $\sigma$. The professor proves that the subset topology of $K$ is coarser than the topology when the number of vertices are infinite. But they are equivalent when the number of vertices are finite ($\mathcal{T}_{1}\subseteq\mathcal{T}_{2}$ and $\mathcal{T}_{2}\subseteq\mathcal{T}_{1}$)
	Now we review abelian groups.
	\begin{definition}
Let G be an abelian group. The set $\{g_{\alpha}\}$ such that every $g\in G$ is
$$
g = \sum n_{\alpha}g_{\alpha},  \text{ finitely many nonzero } n_{\alpha}
$$
is a generator system of $G$. If the coeffiecents are unique then $\{g_{\alpha}\}$ basis.
	\end{definition}
If $G$ has a basis it is called a free abelian group. We can have abelian groups that do not have a basis. The cardinality of the basis is the rank of $\sigma$. These groups are convienent to construct homomorphisms with. We prove a key fact about homomorphsims on these groups in the exercises.
\begin{theorem}
Let $G$ be a finitely generated free Abelian Group. We define $T = \{\text{Elements of finite order}\}$ (T is a subroup of G). The following properties hold:
\begin{itemize}
\item There exists $H$ a free abelian subroup of $G$ such that $G = 
T\oplus  H$
\item There exist some finite cyclic groups $T_{1},...,T_{k}$ with orders $t_{1},...,t_{k}$ respectively such that $T =  T_{1}\oplus...\oplus T_{k}$ and $t_{1}$ divides $t_{2}$, $t_{2}$ divdes $t_{3}$,..., and $t_{k-1}$ divides $t_{k}$ 
\item rank $H$ and $t_{1},...,t_{k}$ are uniquely determined by $G$ 
\end{itemize}
\end{theorem}
$T$ is the torsion subgroup, $t_{1},...,t_{k}$ are the torsion coefficients of $G$ and rank $H$ is the \textit{Betti Number} of $G$
In summary, a finitely generated abelian group always decomposes nicely into finite and infinite parts
$$
G \cong \mathbb{Z}_{t_{2}}\oplus ... \oplus \mathbb{Z}_{t_{k}}\oplus \mathbb{Z}^{r}
$$
Moving on,	now that we have defined a topology on our simplex, we are ready to endow it with a group structure.
	\newline
	
	Let $\sigma$ be a simplex, consider the set $\{\text{Orderings of Vertices}\}$. 
	\begin{itemize}
	\item We define an equivalence class and identify two orderings if they differ by an even permutation (even permutations decompose into an even number of transpositions)
	\item This partitions the set into two equivalence classes
	\item Each of these classes is called an \textit{orientation} of $\sigma$
	\item 0-simplexes only have one orientation
	\end{itemize}
	\begin{definition}
		A simplex with a chosen orientation is called a \textit{oriented simplex }
		\end{definition}
$[v_{0}, ..., v_{p}]$ denotes the simplex $v_{0}, ..., v_{p}$ with the orientation given by $v_{0}< ...< v_{p}$
\begin{figure}[H]
	\centering
	\includegraphics[width=0.6\textwidth]{figures/orientedsimplices.png}
	\label{fig:my_diagram}
\end{figure}

		\begin{exercise}
			Let $K$ be a simplicial complex, prove that the topology of $|K|$
			which we defined during the lesson is actually a topology.
		\end{exercise}
		
		\solution{
			Recall the toplogy we defined in class for $|K|$, which was finer than the topology it inherits as a subset of $\mathbb{R}^{n}$. $|K| = \bigcup\limits_{\sigma \in K} \sigma$
			\newline
			1) The topology of each simplex $\sigma$ is induced by $\mathbb{R}^{n}$
			\newline
			2)$A\subseteq |K|$ is closed iff $A\cap\sigma$ is closed for all $\sigma\in K$
		}
		\newline 
		\newline
		Let $A$ = $\emptyset$. Then $\{\emptyset\}\cap \sigma = \emptyset\ \forall \sigma\in K$. But $\sigma$ inherits a subspace topology from $\mathbb{R}^{n}$ so $\emptyset$ is closed for all $\sigma \in K$. Now let $A = |K|$. We have $|K|\cap \sigma = \sigma \forall \ \sigma \in K$. Because $\sigma$ inherits the subspace topology, $\sigma$ is closed in $\sigma$ for all sigma.
		\newline
		\newline
		Now we check closure under finite unions. Let $A = \bigcup\limits_{i \in I} A_{i}$ be a union of closed sets. $(|I|\in\mathbb{N})$. Then we have $$\big(\bigcup\limits_{i \in I} A_{i}\big)\cap\sigma=(A_{1}\cap \sigma)\cup...\cup(A_{k}\cap\sigma)$$
		Because each $A_{i}$ is closed, all $(A_{i}\cap\sigma)$ must be closed by (2). And a finite union of closed sets is closed.
		\newline
		\newline
		Now let $A = \bigcap\limits_{i \in I} A_{i}$ be an intersection of closed sets. Then we have
		$$\big(\bigcap\limits_{i \in I} A_{i}\big)\cap\sigma = (A_{1}\cap\sigma)\cap...\cap(A_{k}\cap\sigma)\cap...$$
		Each $(A_{1}\cap\sigma)$is closed by (2). And intersections of closed sets are closed.			\begin{exercise}
		Prove that the symmetric group $S_{n}$ is generated by transpositions of type $(j,j+ 1)$.
		\end{exercise}
		
		\solution{
			Recall that $S_{n}$ is the set of all permutations of the set $\{1,...,n\}$. So $\pi\in S_{n}$ maps $\pi:\{1,...,n\}\to\{1,...,n\}$ where the group operation is composition. Let $\pi$ be an arbitrary element of $S_{n}$. We would like to show that 
			$$
			\pi \;=\; \Pi_j (j,j+1)^{n_{j}}, \quad \text{where}\ 
			(j,j+1)^{n_{j}} \;=\; 
			\underbrace{(j,j+1)\circ (j,j+1)\circ \cdots \circ (j,j+1)}_{n_j \text{ times}}
			$$ But we also know that any $\pi$ can be decomposed into a composition of transpositions (2-cycles) $$\pi = (a_1\,a_2\,\dots\,a_k) \;=\; (a_1\,a_k)\circ(a_1\,a_{k-1})\circ\cdots\circ(a_1\,a_2)$$
		}
		But each transposition can be decomposed into compositions of our generators $\{(j,j+1)\}$.
		$$
		(a_{m}a_{n}) = (j ,k) = (j,j+1)\circ(j+1,j+2)\circ...(k-1,k)\circ(k-2,k-1)\circ...\circ(j+1,j+2)\circ(j,j+1)
		$$
		Putting this into the previous decomposition and using the fact that the symmetric group is abelian we conclude that every element of the permuation group can be expressed in the form
		$$
		 \pi \;=\; \Pi_j (j,j+1)^{n_{j}}
		$$
			\begin{exercise}
			Let $F$ be a free abelian group with basis $(e_{\alpha})_{\alpha\in J}$ and $A$ be an
			abelian group. Prove that for each system of vectors $(g_{\alpha})_{\alpha\in J}$ of $A$ there exists
			exactly one homomorphism $h : F \to A$ such that $h(e_{\alpha}) = g_{\alpha}$ for all $\alpha\in J$.
		\end{exercise}
		
		\solution{
			$F$ is a free abelian group, so every element $f\in F$ can be written as
			$$
			f = \sum n_{\alpha}e_{\alpha}\text{with finitely many } n_{\alpha} \neq 0
			$$
			Before showing uniqueness we construct the homomorphism $h$. Define $h:F\to A$ such that
			$$
			h(f) = h(\sum n_{\alpha}e_{\alpha}) = \sum n_{\alpha}h(e_{\alpha}) = \sum n_{\alpha} g_{\alpha}
			$$
			It is the same $n_{\alpha}$ from before so the sum is well defined. Let us choose $x,y\in F$ so that
			$$
			h(x) +h(y) = h(\sum m_{\alpha} e_{\alpha})+h(\sum n_{\alpha}e_{\alpha}) = \sum 
			(n_{\alpha}+m_{\alpha})g_{\alpha} = h(\sum(m_{\alpha}+n_{\alpha}))
			=h(x+y)$$ 
					}
					So we see it is a homomorphism. Let us assume $h^{\prime}(e_{\alpha}) = g_{\alpha}$is another homomorphism and let $x$ be an arbitrary element in $F$
					Thus we have
					$$
					h(x) = h(\sum m_{\alpha}e_{\alpha}) = \sum m_{\alpha}g_{\alpha} =  h^{\prime}(\sum m_{\alpha}e_{\alpha}) =  h^{\prime}(x)
					$$
					So our homomorphism $f$ is unique.
			\begin{exercise}
		Let $G$ be isomorphic to $\mathbb{Z}_{28} \oplus\mathbb{Z}_{42} \oplus\mathbb{Z}_{100}\oplus\mathbb{Z} \oplus\mathbb{Z}_{154}\oplus\mathbb{Z} \oplus\mathbb{Z}_{99}$.
		Compute the torsion coefficieients and the Betti number of $G$.
		\end{exercise}
		
		\solution{
			We see that the group has two copies of $\mathbb{Z}$ so the betty number is 2. The strategy to get the torsion coefficients is described in the next problem. They are $2,14,462,13860$. We also find a factor of one in our decomposition but that gives the trivial group so we ignore it.
		}
			\begin{exercise}
			\textit{(Extra)} Compute the torsion coefficients  of $\mathbb{Z}_{30} \oplus\mathbb{Z}_{18} \oplus\mathbb{Z}_{75}$.
		\end{exercise}
		
		\solution{
			We do a prime factorization, $30 = 2\cdot 3\cdot 5$, $18 = 2\cdot 3^{2}$ and $75 = 3\cdot 5^{2}$. If we make vectors of the powers each of the prime factors in the original numbers (for example $18\equiv[1,2,0]$), we can then stack all the vectors and rearrange them in ascending order. Then we read the new numbers powers of the exponents down the columns. So we get the torsion coefficients $2^{0}\cdot 3^{1}\cdot 5^{0} = 3$, $2^{1}\cdot 3^{1}\cdot 5^{1} = 30$ and $2^{1}\cdot 3^{2}\cdot 5^{2} = 450$ 
		}. As a check one can see that 3 divides 30 and 30 divedes 450. Also $3\cdot 30\cdot 450 = 30\cdot 18\cdot 75$
			\begin{exercise}
			\textit{(Extra)} Compute the torsion coefficients  of $\mathbb{Z}_{50} \oplus\mathbb{Z}_{36} \oplus\mathbb{Z}_{2}\oplus\mathbb{Z}_{5}$.
		\end{exercise}
		
		\solution{
			The coefficients are $2,10,900$.
		}
	
		\lecture{3}{Group of Oriented P Chains}
\begin{definition}
	Let $X =  \{x_{\alpha}\}$ be a set. We define
$$
G = \{\sum n_{\alpha}x_{\alpha}|n_{\alpha}\in\mathbb{Z}, n_{\alpha}\ \text{almost all 0}\}
$$
We define the group operation to be addition where
$$
\sum n_{\alpha}x_{\alpha} + \sum m_{\alpha}x_{\alpha} = \sum (n_{\alpha}+m_{\alpha})x_{\alpha}
$$
This is a free abelian group with basis $X\subseteq G$
\end{definition}
\begin{definition}
Let $K$ be a simplicial complex.
\begin{align}
\scriptstyle{
C_{p}(K)=\frac{\text{Free Abelian Group generatd by oriented p-simplices of $K$}}{\text{the subgroup generated by $\sigma+\sigma^{\prime}$ such that $\sigma$ and $\sigma^{\prime}$ are the same simplex with opposite orientation}}}
\end{align}
$C_{p}(K)$ is the group of oriented p-chains.
\end{definition}
In plain english we take a simplicial complex and pickout all the simplices with codimension k. Then we treat the simplices that are the same with opposite orientations as one.
\begin{figure}[H]
	\centering
	\includegraphics[width=0.6\textwidth]{figures/pchains.pdf}
	\label{fig:my_diagram}
\end{figure}
If we just choose an orientation for each p simplex then we have a basis for $C_{p}(K)$. There are arguments from general group theory that imply the existence of a homomorphism. We skip these details but note that a homomorphism $f$ would map our free abelian group like
$$
f\big(\sum n_{\alpha}x_{\alpha}\big) = \sum n_{\alpha} f(x\alpha)
$$
on the basis elements. We use this to define the boundary map.
\begin{definition}
Let $K$ be a simplicial complex and we define the following function.

$$\partial_{p}:\{\text{Oriented p-simplices of K}\}\to C_{p-1}(K)$$
$$\partial_{p}([v_0,...,v_p]) = \sum_{i=0}^{p}(-1)^{i}[v_0,...,\hat{v}_{i},...v_p]$$
where $\hat{v}_{i}$ indicates we remove that vertex from the simplex. This is called the boundary operator
\end{definition}
Notice that $(-1)^{i}$ ensures that the orientations of the faces are consistent. This will be necessary to ensure $\partial_{p-1}\circ\partial_{p}= 0$. In the notes it is shown that this is in fact a homomorphism and preserves group structure. Now time for an example.
\begin{figure}[H]
	\centering
	\includegraphics[width=0.6\textwidth]{figures/boundary_operator.png}
	\label{fig:my_diagram}
\end{figure}
	\begin{exercise}
Show that $[\partial_{p},\pi] = 0$ with $\pi\in S_{p}$. This is, show that the boundary operator commutes with permutation when acting on elements of $C_{p}(K)$
\end{exercise}
\lecture{4}{Boundary Operator Properties and Homology Group}
Now that we have defined the boundary operator, let us prove the property $\partial_{p-1}\circ\partial_{p} = 0$.

$$\partial_{p-1}\circ\partial_{p}[v_{0},...,v_{p}]= \partial_{p-1}\sum_{i=0}^{p}(-1)^{i}[v_{0},...,\hat{v}_{i},...,v_{p}] = \sum_{i=0}^{p}(-1)^{i}\partial_{p-1}[v_{0},...,\hat{v}_{i},...,v_{p}] =$$
\textcolor{red}{ask professor about the proof and how to break up the sums ...}
$$
= \sum_{i>j}(-1)^{i+j}[v_{0},...,\hat{v}_{j},...,\hat{v}_{i},...,v_{p}]-\sum_{j>i}(-1)^{i+j}[v_{0},...,\hat{v}_{i},...,\hat{v}_{j},...,v_{p}] = 0$$
The geometric idea is that the boundary of a boundary is nothing.
\begin{definition}
We have two preliminary definitions.
\begin{itemize}
\item The kernel of $\partial_{p}:C_{p}(K)\to C_{p-1}(K)$ is called the group of p-cycles and is denoted $Z_{p}(K)$
\item The image of $\partial_{p+1}:C_{p+1}(K)\mapsto C_{p}(K)$ is called the group of p-boundaries and is denoted $B_{p}(K)$ 
\end{itemize}
Not that $\partial_{p-1}\circ\partial_{p} = 0 \Rightarrow B_{p}(K) \subseteq Z_{p}(K)$. We have
\begin{center}
$H_{p}(K)\equiv Z_{p}(K)/B_{p}(K)$ is called the $p^{th}$ homology group.
\end{center}
\end{definition}
Now let us do some example calculations.
\begin{figure}[H]
	\centering
	\includegraphics[width=1.0\textwidth]{figures/homcalcpg1.pdf}
	\label{fig:my_diagram}
\end{figure}
\begin{figure}[H]
	\centering
	\includegraphics[width=1.0\textwidth]{figures/homcalcpg2.pdf}
	\label{fig:my_diagram}
\end{figure}
The Homology group $H_{p}(K)$ detects holes p dimensional holes in your space from the triangulation of the manifold. We can see that in the two examples we just did
\intermezzo{2}{Review of Some Relevant Definitions from Munkes Topology Chapter 3}
% ====================================
% --- CONNECTEDNESS AND COMPACTNESS ---
% ====================================

\begin{definition}[Connected Space]
	A topological space $X$ is said to be \emph{connected} if it cannot be written as the union of two disjoint nonempty open subsets.  
	Equivalently, there do not exist disjoint nonempty open sets $U,V \subseteq X$ with $X = U \cup V$.
\end{definition}

\begin{definition}[Path Connected Space]
	A space $X$ is \emph{path connected} if for every pair of points $x,y \in X$ there exists a continuous map
	\[
	\gamma : [0,1] \to X
	\]
	such that $\gamma(0) = x$ and $\gamma(1) = y$.  
	The map $\gamma$ is called a \emph{path} from $x$ to $y$.
\end{definition}

\begin{definition}[Components]
	Let $X$ be a topological space.
	\begin{itemize}
		\item A subset $C \subseteq X$ is \emph{connected} if it is connected as a subspace.
		\item A \emph{component} (or \emph{connected component}) of $X$ is a maximal connected subset of $X$, i.e. a connected subset not properly contained in any larger connected subset.
	\end{itemize}
	The components of $X$ form a partition of $X$, and each component is closed in $X$.
\end{definition}

\begin{definition}[Compact Space ]
	A topological space $X$ is \emph{compact} if every open cover of $X$ admits a finite subcover; that is,  
	whenever $\{\,U_\alpha\,\}_{\alpha \in A}$ is a collection of open subsets with $X = \bigcup_{\alpha \in A} U_\alpha$,  
	there exist finitely many indices $\alpha_1,\dots,\alpha_n$ such that
	\[
	X = U_{\alpha_1} \cup \cdots \cup U_{\alpha_n}.
	\]
\end{definition}
\begin{example}[Connected but not Path Connected: The Topologist's Sine Curve]
	Consider the subset
	\[
	S = \bigl\{ (x, \sin(1/x)) : 0 < x \le 1 \bigr\} \cup \bigl(\{0\} \times [-1,1]\bigr)
	\subset \mathbb{R}^2.
	\]
	\begin{enumerate}
		\item The set $\{(x, \sin(1/x)) : 0 < x \le 1\}$ is the graph of $\sin(1/x)$, which oscillates infinitely often as $x \to 0^+$.
		\item The vertical segment $\{0\} \times [-1,1]$ is added to make the set closed in $\mathbb{R}^2$.
	\end{enumerate}
	Then:
	\begin{itemize}
		\item $S$ is \emph{connected} — any attempt to separate $S$ into disjoint nonempty open subsets fails, because the oscillations of $\sin(1/x)$ accumulate densely along the vertical segment.
		\item $S$ is \emph{not path connected} — there is no continuous path in $S$ joining a point on the oscillating part (where $x>0$) to a point on the segment $\{0\}\times[-1,1]$. Any such path would force a limit of $\sin(1/x)$ as $x\to 0^+$, which does not exist.
	\end{itemize}
	Thus, $S$ provides a standard example of a space that is connected but not path connected.
\end{example}
\begin{figure}[H]
	\centering
	\includegraphics[width=0.6\textwidth]{figures/topologistsinecurve.png}
	\label{fig:my_diagram}
\end{figure}
\lecture{5}{Homology Group Structure and Equivalent Homology Group}
\begin{definition}
Let $K$ be a simplicial complex with $v\in K^{(0)}$
We define the star of $v$
$$
\text{st}(v) = \cup_{\sigma\in K}\text{Int}(\sigma)\ \text{such that}\ v\ \text{is a vertex of }\sigma
$$
\end{definition}
\begin{figure}[H]
	\centering
	\includegraphics[width=0.7\linewidth]{figures/stv.png}
	\label{fig:note-oct-8-2025}
\end{figure}
We can take its closure. We can also see that st($v$) is an open set of $|K|$ so that $|K|/\text{st}(v)$ is closed in the topology.
\begin{theorem}
Let $K$ be a simplicial complex and $H_{0}(K)$ is a free abelian group. If $\{v_{\alpha}\}\subseteq K^{(0)}$ such that each connected component of $|K|$ contains exactly one element of $\{v_{\alpha}\}$, then the equivalence classes of $\{v_{\alpha}\}$ give a basis of $H_{0}$(K)
\end{theorem}
This is an extremely confusing definition. The point is that $H_{0}(K)$ just counts the number of connected components. It would make more sense to consider a space of multiple components, then show that each component is generated by one vertex using properties of connectedness. But I omit the proof from these notes. Using this idea we can create a new definition from our star definition.
\begin{definition}
$$
C_{v} = \cup_{w\sim v}\  \text{st}(w)
$$
are the connected components of $|K|$
\begin{itemize}
\item $C_{v}$ are open in the topology on $|K|$
\item $C_{v}$ are (path) connected
\end{itemize}
\end{definition}
I dont put the proof here but it should be clear that if there is an edge between $v$ and $w$ That if we take the union of all these stars it gives us the connected components of $|K|$. Now let $\{v_{\alpha}\}$ be a set of vertices such there is one vertex from each connected component. The boundary map is trivially 0, $\partial_{0}C_{0}\to C_{-1} = 0$. It follows that $C_{0}(K) = Z_{0}(K)$. Now let $w\in K^{(0)}$. Then there exists $\alpha$ and $d\in C_{1}(K)$ such that $w-v_{\alpha}=\partial d $. In plain english, if $w$ is a vertex then there is a one cycle $d$ such that $w-\v_{\alpha}$ is the boundary of $d.$ The proof is a calculation
\begin{figure}[H]
	\centering
	\includegraphics[width=0.9\linewidth]{figures/connectedproof.png}
	\label{fig:note-oct-8-2025}
\end{figure}
So now we can undrestand $H_{0}(K)$ a little better. If two vertices are connected by edges then they are in the same $H_{0}(K)$ equivalent class. But that is only the case if they are connected. So the $H_{0}(K)$ counts the number of connected components of the manifold. The $\{v_{\alpha}\}$ form a basis for $H_{0}(K)$. So all elements of this group are linear combinations. The proof uses definiton of a basis and the boundary map. I omit it but we can use this idea to define a new function.
\begin{definition}
Define a homomorphism $\varepsilon: C_{0}(K)\to\mathbb{Z}$ such that $\varepsilon(v)= 1\ \forall v\in C_{0}(K)$. That is $\varepsilon(\sum n_{\alpha}v_{\alpha}) = \sum{n_{\alpha}}$
We call this function the augmentation map.
\end{definition}
From this definition it is clear that $\varepsilon(\partial [v,w]) = 1 - 1 = 0 \Rightarrow \varepsilon \circ \partial_{1} = 0$. Now we can define the reduced homology groups. First recall $H_{0}(K)=C_{0}(K)/\Im{\partial_{1}}$
\begin{definition}
Attach the augmentation map $\varepsilon:C_{0}(K)\to\mathbb{Z}$ to the chain complex to form the
\emph{augmented} complex
\[
\cdots \xrightarrow{\partial_{2}} C_{1}(K)
\xrightarrow{\partial_{1}} C_{0}(K)
\xrightarrow{\ \varepsilon\ } \mathbb{Z} \to 0 .
\]
The \emph{reduced homology groups} $\widetilde H_n(K)$ are the homology groups of this augmented complex; equivalently,
\[
\widetilde H_n(K)=
\begin{cases}
	\ker(\partial_n)\big/ \operatorname{im}(\partial_{n+1}), & n\ge 1,\\[4pt]
	\ker(\varepsilon)\big/ \operatorname{im}(\partial_{1}), & n=0.
\end{cases}
\]
In particular, for $n\ge 1$ we have $\widetilde H_n(K)\cong H_n(K)$, while
$\widetilde H_0(K)=\ker(\varepsilon)/\operatorname{im}(\partial_1)$.
\end{definition}
We can contrast this to Homology groups:\[
\text{Homology Groups:}\quad
\cdots \xrightarrow{\partial_{2}} C_{1}(K)
\xrightarrow{\partial_{1}} C_{0}(K)
\xrightarrow{\;0\;} 0 \xrightarrow{\;0\;} 0
\]
This lecture concluded with the following theorem
\begin{theorem}
$\tilde{H}_{0}(K)$ is free and $H_{0}(K)\simeq\tilde{H}_{0}(K)\oplus \mathbb{Z}$. If $K$ is connected than $\tilde{H}_{0}(K) = 0$
\end{theorem}
There is a proof in the notes but we already know $H_{0}(K) = \mathbb{Z}$ if it is connected. So of course this reduced homolgy group must be 0.
	\begin{exercise}
Let $K$ be a simplicial complex and $Y$ be a topological space.  
Prove that $f : |K| \to Y$ is a continuous function if and only if $f|_{\sigma}$ is continuous for all $\sigma \in K$.
\end{exercise}
\solution{
$(\Rightarrow)$ We use theorem 18.2 (d) of Munkres Topology for constructing continuous functions. If $f:|K|\to Y$ is continuous and $\sigma \subset |K|$, then $f|_{\sigma}: \sigma \to Y$ is also continuous. This is true for all the $\sigma$ in $|K|$, so the forwards statement is true.
\newline

$(\Leftarrow)$ Now we assume that for all $\sigma_i\in |K|$ that $f|_{\sigma_i}:\sigma_i\to Y$ are continuous functions. So for each closed set $C \subseteq Y$ we have that $(f|_{\sigma_i})^{-1}(C)$ is closed in $\sigma_i$. Each $\sigma_i$ has the toplogy it inherits from $\mathbb{R}^{n}$ so $(f|_{\sigma_i})^{-1}(C)$ is closed in this topology. But because $(f|_{\sigma_i})^{-1}(C)$ is closed and $\sigma_i$ is closed in the toplogy induced by $\mathbb{R}^{n}$, we have that $ (f|_{\sigma_i})^{-1}(C)\cap \sigma_i = (f|_{\sigma_i})^{-1}(C)$. Because this holds for all the possible $\sigma_i\in K$ , $(f|_{\sigma_i})^{-1}(C)$ will always be  closed in the topology of $|K|$. Because $C\subseteq Y$ was a closed set, it follows that $f:|K|\to Y$ is continuous.

}
\begin{exercise}
\begin{enumerate}
	\item Let $x$ be a point of the simplicial complex $v_0 v_1 \cdots v_p$, then we have that
	\[
	x = \sum_{i=0}^p t_i v_i
	\]
	with $t_i \geq 0$ and $\sum_{i=0}^p t_i = 1$.  
	Prove that the coefficients $t_i$ are uniquely determined by $x$.  
	The real numbers $t_i$ are called \textit{barycentric coordinates} of $x$.
	
	\item Let $K$ be a simplicial complex. Prove that each $x \in |K|$ is contained in the interior of exactly one simplex of $K$.  
	Then fix $v \in K^{(0)}$ and define
	\[
	t_v : |K| \to \mathbb{R}
	\]
	as the function associating $x \in |K|$ with the barycentric coordinate of $x$ with respect to $v$ if $x$ is contained in the interior of a simplex of vertex $v$, otherwise we set $t_v(x) = 0$.  
	Prove that $t_v$ is continuous.
\end{enumerate}
\solution{
\begin{enumerate}
\item Consider two points in $|K|$, $x$ and $x^{\prime}$. Let us assume $x = x^{\prime}$. Thus we have
$$
\sum_{i=0}^{p}t_{i}v_{i} = \sum_{i=0}^{p}t^{\prime}_{i}v_{i}, \quad \sum_{i=0}^{p}t_{i}=\sum_{i=0}^{p}t_{i}^{\prime} = 1,\quad t_{i},t_{i}^{\prime}\geq 0
$$
We can subtract a vertex $v_{j}$ from each point to get a set of linearly independent vectors. Now we have
$$
\vec{x} = \sum_{i=0}^{p}t_{i}(v_{i}-v_{j}),\quad \vec{x}^{\prime}= \sum_{i=0}^{p}t^{\prime}_{i}(v_{i}-v_{j})
$$
Because $\vec{x} = \vec{x}^{\prime}$ we have
$$
\vec{0} = \vec{x}-\vec{x}^{\prime} = \sum_{i=0}^{p}(t_{i}-t_{i}^{\prime})\vec{v}_{ij}
$$
The basis is linearly independent so $t_{i} = t_{i}^{\prime}\ \forall i$. So the coefficients $t_{i}$ uniquely determine the point $x$. 
\item Now let $x$ belong to the interior of two simplexes of $\sigma$ and $\tau$ of the simplicial complex $K$. $\sigma \cap \tau \neq \emptyset \Rightarrow \sigma \cap \tau $ is a face common to both $\sigma$ and $\tau$. But this cannot be a proper face because this would mean $x\in\partial \sigma$ or $\partial\tau$. So we can conclude 
$$
\sigma\cap\tau = \sigma\  \text{and}\  \sigma\cap\tau = \tau \Rightarrow \sigma = \tau
$$
So $x$ can only belong to the interior of one simplex of $K$
\newline 

Now we fix $v \in K^{(0)}$ and define $t_v : |K| \to \mathbb{R}$ by
\[
t_v(x) =
\begin{cases}
	t_i, & \text{if } v = v_i \in \sigma \text{ and } x \in \operatorname{int}\sigma, \\[4pt]
	0,   & \text{if } v \notin \sigma.
\end{cases}
\]
Choose $\sigma = \langle v_0, v_1, \dots, v_p \rangle$. 
If $v \notin \sigma$, then $(t_v|_{\sigma})(x) = 0$ for all $x$ in the interior of $\sigma$.
This is a constant map, hence continuous.  

Now fix $v_0 \in \sigma$ and consider $x, x' \in \sigma$.  
Define the vectors
\[
\vec{x} = x - v_0 = \sum_{i=1}^{p} t_i (v_i - v_0) = \tilde{B}\,\vec{t}, 
\qquad
\vec{x}' = x' - v_0 = \sum_{i=1}^{p} t_i' (v_i - v_0) = \tilde{B}\,\vec{t}',
\]
where 
\[
\tilde{B} = [\,v_1 - v_0 \ \ v_2 - v_0 \ \ \cdots \ \ v_p - v_0\,],
\quad
\vec{t} = (t_1, \dots, t_p)^{\mathsf T}.
\]
Because the vertices of $\sigma$ are geometrically independent, the columns of $\tilde{B}$ are linearly independent.
Thus $\tilde{B}$ is invertible and we have
\[
\vec{t} - \vec{t}' = \tilde{B}^{-1} (\vec{x} - \vec{x}').
\]
Taking norms gives
\[
\|\vec{t} - \vec{t}'\| \le \|\tilde{B}^{-1}\|\,\|\vec{x} - \vec{x}'\|.
\]
Because $t_{0} = 1 - \sum_{i=1}^{p}t_{i}$ we have
\[
|t_0 - t_0'|
= \Big|\sum_{i=1}^p (t_i' - t_i)\Big|
\le \sum_{i=1}^p |t_i' - t_i|
= (1,\dots,1) \cdot (|t_1 - t_1'|, \dots, |t_p - t_p'|)
\le \sqrt{p}\,\|\vec{t} - \vec{t}'\|
\le \sqrt{p}\,\|\tilde{B}^{-1}\|\,\|\vec{x} - \vec{x}'\|.
\]
Finally, given $\varepsilon > 0$, choose
\[
\delta = \frac{\varepsilon}{\sqrt{p}\,\|\tilde{B}^{-1}\|}.
\]
Then, whenever $\|\vec{x} - \vec{x}'\| < \delta$, we have
\[
|t_{v_0}(x) - t_{v_0}(x')|
= |t_0 - t_0'|
\le \sqrt{p}\,\|\tilde{B}^{-1}\|\,\|\vec{x} - \vec{x}'\|
< \sqrt{p}\,\|\tilde{B}^{-1}\| \cdot 
\frac{\varepsilon}{\sqrt{p}\,\|\tilde{B}^{-1}\|}
= \varepsilon.
\]
This holds for all $v_{i}$ so $t_{v}$ is continuous on $\sigma$.
Because the function restricted to $\sigma$ is continuous for all $\sigma\in K$ we use the result of the first exercise to conclude that the function is continuous on the whole space.
 \end{enumerate}
}
	\begin{exercise}
	Let $K$ be a simplicial complex. Prove that:
	\begin{enumerate}
		\item $|K|$ is Hausdorff;
		\item in $|K|$ path-connected components and connected components coincide;
		\item if $K$ is finite, $|K|$ is compact.
	\end{enumerate}
		\end{exercise}
	\end{exercise}
	\solution{	
\begin{enumerate}
\item Let $\sigma$ and $\sigma^{\prime}$ be disctinct simplexes in $|K|$. First, let $x,y$ be distinct points in $\sigma$. Because $\sigma$ inherits the topology of $\mathbb{R}^n$ and $\mathbb{R}^{n}$ is Hausdorff, there exists $U$ and $V$ disjoint sets in $\sigma$ such that $x\in U$ and $y\in V$. Now let $x\in \sigma$ and $y\in\sigma^{\prime}$. If $\sigma\cap\sigma^{\prime}=\emptyset$ then the Hausdorff criteria automatically follows. But if $\sigma\cap\sigma^{\prime}\neq\emptyset $ then either $x$ or $y$ or both are in $\sigma\cap\sigma^{\prime}$. In that case we invoke the Hausdorff property of $\mathbb{R}^{n}$ inherited by $\sigma\cap\sigma^{\prime}$. In all cases $|K|$ is Hausdorff.
\item Path connected always imply connected. So we only need to prove that the connected components of $|K|$ are also path connected. Each simplex $\sigma \subset |K|$ is convex in $\mathbb{R}^n$, hence path-connected.
If two simplices $\sigma$ and $\tau$ intersect, then $\sigma \cap \tau$ is a common face,
which is also convex and therefore path-connected.
Thus, $\sigma \cup \tau$ is path-connected.

Now note that any two simplices of $K$ that intersect can be joined by a path through their
common face.  
If we can move from one simplex to another through a chain of intersecting simplices,
then any two points in their union can be connected by a continuous path in $|K|$.
Each connected component of $|K|$ is exactly such a union,
and therefore is path-connected.
Hence the connected components and path-connected components of $|K|$ coincide.

\item Because the polytope $|K|$ is embedded in $\mathbb{R}^{n}$ we use the Heine-Borel theorem which states that compactness is equivalent to boundedness and closedness. Because the vertices $\{v\}$ in $|K|$ are geometrically independent, we subtract all vertices from $\vec{0}$ to get $\{\vec{v}\}$. Let $R = \max{||\vec{v}||}$ such that $\vec{v}\in\{\vec{v}\}$. Then we can bound the Polytope in a ball $B_{R}(0)$. So $|K|$ is bounded. Because each $\sigma$ is closed in $\mathbb{R}^{n}$, $|K|=\cup_{\sigma\in K}\sigma$ is also closed. It follows that $|K|$ is compact.
\end{enumerate}
\lecture{6}{Homology Group of A Cone and more Reduced Homology Group Properties}
	 Given a simplicial complex $|K|$, we can define an operation to add another vertex and create a  new simplicial complex.

\begin{definition}
	We define the cone on $K$ with vertex $w$ as a point $w\in\mathbb{R}^{n}$ that  intersects $|K|$ in at most one point. 
	$$
	w*K = K\cup\{wa_{1}...a_{p}|a_{1}...a_{p}\in K\}\cup\{w\}
	$$
\end{definition}
One can verify that this new structure is also a simplicial complex. The point $w$ cannot be collinear with any of the 1 faces of $|K|$.
\begin{definition}
	Let $w*K$ be a cone. We define a homomorphism
	$$
	C_{p}(K)\to C_{p+1}(w*K)
	$$
	$$
	[a_{0},...,a_{p}]\to[w,a_{0},...,a_{p}]
	$$
\end{definition}
Now we have a nice interesting theorem.
\begin{theorem}
	$\tilde{H}_{p}(w*K) = 0\ \forall\ p$. We call this property acyclic. 
\end{theorem}
Adding a cone to even a nonconnected subspace makes in contractible to a point. And removes all holes. Like a torus, adding a cone will make the entire space connected and remove the holes. The proof is a calculation again. $p=0$ trivial because $w*K$ is connected. Then you do a calculation for $p>1$. The proof is carried out in the lecture notes.
\begin{theorem}
	Let $\sigma$ be an n-simplex.  We have $K_{\sigma}$, the simplicial complex built from $\sigma$ (filled in polyhedron) is acyclic. For $n>0$ we define
	$$
	\Sigma^{n-1} = \{\text{proper faces of $\sigma$}\}
	$$
	We have that 
	\begin{itemize}
		\item $\tilde{H}_{n-1}(\Sigma^{n-1})\simeq \mathbb{Z}$
		\item $\tilde{H}_{i}(\Sigma^{n-1})=0\ \forall i \neq n-1$
	\end{itemize}
	If we give an orientation to $\sigma$ then $\partial \sigma$ generates $\tilde{H}_{n-1}(\Sigma^{n-1})$
\end{theorem}
	}
	\begin{figure}[H]
		\centering
		\includegraphics[width=0.7\textwidth]{figures/homework3.pdf}
		\label{fig:my_diagram}
\end{figure}
\begin{exercise}
Do the above exercise
\end{exercise}
	
	\lecture{7}{Beginning of Singular Homology and its Topological Invariance}
	We use the notation $e_{0},...,e_{p}$ is the standard basis for $\mathbb{R}^{p+1}$. We define $\Delta_{p}\equiv e_{0}...e_{p}$ as the standard p-simplex
	\[
	\Delta_p = \{ (t_0, \ldots, t_p) \in \mathbb{R}^{p+1} \mid t_i \ge 0, \ \sum t_i = 1 \}.
	\]
\begin{figure}[H]
	\centering
	\includegraphics[width=0.5\textwidth]{figures/polytope.png}
	\label{fig:my_diagram}
\end{figure}
\begin{definition}
	Let $X$ be a topological space. A singular $p-$simplex is a continuous map
	$$
	\sigma:\Delta_{p} \to X
	$$
\end{definition}
The only condition on $\sigma$ is continuity. In simplicial homology we focused on the combinatorial structure of the building blocks of the polytope. Now we are focusing on the whole space, studying all possible continuous maps from the polytope to a topological space $X$. 
\begin{definition}
The free abelian group generated by the singular p-simplices is called the singular p-chain group $S_{p}(X)$.
\end{definition}
The members of this group are the maps $\sigma_{i}$ and $X$ is the target space, the space we wish to study. Let ${a_{0}},...,a_{p}\subseteq \mathbb{R}^{n}$, not necessarily geometricall independent. We define
$$
\varphi (a_{0},...,a_{p}):\Delta_{P}\to\mathbb{R}^{n}
$$
$$
\sum t_{i}e_{i} \to \sum t_{i}a_{i}
$$
$\varphi (a_{0},...,a_{p})$ is a p simplex of $R^{n}$ (continuous). We have $\varphi (e_{0},...,e_{P}) \equiv \text{Id}_{\Delta_{P}}.$ We also have $\varphi (e_{0},..,\hat{e}_{i},.,e_{P}):\Delta_{p-1}\to\Delta_{P}$. As an example
\begin{figure}[H]
	\centering
	\includegraphics[width=0.5\textwidth]{figures/examplemap1.png}
	\label{fig:my_diagram}
\end{figure}
We have a basis for $S_{P}(X)$, the set $\{\sigma|\text{\ singular p simplices}\} = \mathcal{S}$ Any function $\mathcal{S}\to G \text{ (Abelian Group)}$ can be uniquely extend to homomorphism $S_{P}(X)\to G$ We can now define another boundary operator in this Homology setting.
\begin{definition}
Let $\partial_{P}:S_{P}(X)\to S_{P-1}(X)$ by the homomorphism
$$
\partial_{P}(\sigma)=\sum_{i=0}^{p}(-1)^{i}\sigma \circ \varphi(e_{0},...,\hat{e}_{i},...,e_{p})
$$
where $\sigma\circ\varphi(e_{0},...,\hat{e}_{i},...e_{p})$
\end{definition}
This operator works like the boundary in simplicial homology, $\Im{\partial_{p}}\subseteq \text{Ker}{\partial_{p-1}}$. The proof is analagous to the simplical homology proof. With this definition we define the chain complex.
\begin{definition}
We define the chain complex $\{S_{P}(X),\partial_{P}\}$
\newline
$$
\cdots 
\xrightarrow{\partial_{p+2}} 
S_{p+1}(X) 
\xrightarrow{\partial_{p+1}} 
S_{p}(X) 
\xrightarrow{\partial_{p}} 
S_{p-1}(X) 
\xrightarrow{\partial_{p-1}} 
\cdots 
\xrightarrow{\partial_{1}} 
S_{0}(X) 
\longrightarrow 0
$$
\begin{itemize}
\item $Z_{P}(X)\equiv \text{ker}\partial_{P} \text{ group of (singular) p-cycles}$
\item $B_{P}(X)\equiv \partial_{P+1}(S_{P+1}(X)) \text{ group of (singular) p-boundaries}$
\end{itemize}
\end{definition}
We have $\partial_{P-1}\partial_{P} = 0 \Rightarrow B_{P}(X)\subseteq Z_{P}(X)$. We have that
$$
H_{P}(X) = Z_{P}(X)/B_{P}(X) \equiv \text{ the singular pth homology group}
$$
We would eventually like to prove that these groups are topological invariants, invariants under homeomorphism. To prove this we will need to build up a bit of the algebra of these groups before we can make this claim that singular homology groups are topological invariants.
\begin{definition}
Let $\{A_{p}\},\ p\in\mathbb{Z}$ be a family of abelian groups and $\{\partial_{P}:A_{p}\to A_{p-1}\}$ be homomorphisms such that $\partial_{p}\circ\partial_{p+1}=0$. Then in general we have what is called a chain complex
$$
\cdots 
\xrightarrow{\partial_{p+2}} 
A_{p+1}
\xrightarrow{\partial_{p+1}} 
A_{p}
\xrightarrow{\partial_{p}} 
A_{p-1}
\xrightarrow{\partial_{p-1}} 
\cdots 
\xrightarrow{\partial_{1}} 
A_{0}
\longrightarrow 0
$$
\end{definition}
In this more general setting where we dont care about the topological space $X$, we still have that $\ker{\partial_{P}}\subseteq \Im{\partial_{p+1}}$. From this we have a general homology group for $\{A_{p}, \partial_{P}\}$
$$
H_{P}(A)=\Im{\partial_{p+1}}/ \ker{\partial_{P}}
$$ We can actually define homomorphisms between two chain maps.
\begin{definition}
Let $\varphi = \{C_{p},\partial_{P}\}$ and $\varphi^{\prime} = \{C_{p}^{\prime},\partial_{P}^{\prime}\}$ be chain complexes. A family of homomorphisms $\{\phi_{P}:C_{P}\to C_{P}^{\prime}\}$ such that $\phi_{P-1}\circ\phi_{P}= \partial_{P}^{\prime}\circ\phi_{P}$ is called a chain map from $\varphi$ to $\varphi^{\prime}$
\[
\begin{array}{ccccccccc}
	\cdots & \xrightarrow{\partial_{p+2}} & C_{p+1} & \xrightarrow{\partial_{p+1}} & C_{p} & \xrightarrow{\partial_{p}} & C_{p-1} & \xrightarrow{\partial_{p-1}} & \cdots \xrightarrow{\partial_{1}} C_{0} \longrightarrow 0 \\
	& & \ \downarrow \phi_{p+1} & & \ \downarrow \phi_{p} & & \ \downarrow \phi_{p-1} & & \\
	\cdots & \xrightarrow{\partial_{p+2}'} & C^{\prime}_{p+1} & \xrightarrow{\partial_{p+1}'} & C^{\prime}_{p} & \xrightarrow{\partial_{p}'} & C^{\prime}_{p-1} & \xrightarrow{\partial_{p-1}'} & \cdots \xrightarrow{\partial_{1}'} C^{\prime}_{0} \longrightarrow 0
\end{array}
\]
\end{definition}
With our chain map definition we can study maps between singular homology groups. Now let $\phi:\varepsilon\to\varepsilon^{\prime}$ be chain map. We can define a new function.
\begin{definition}
$$
(\phi_{*})_{p}:H_{p}(\epsilon)\to H_{p}(\epsilon^{\prime})
$$
$$
(\phi_{*})_{p}([c]) = [\phi([c])]
$$ 
and we have a set for all the simplices $\phi_{*} = \{(\phi_{*})_{p}\}$
\end{definition}
This map now maps between the equivalence classes, but $\phi$ just mapped between teh chain groups. $\phi_{*}$ is well defined. Let $c,c^{\prime}\in\ker{\partial_{p}}\ \text{such that} [c] = [c^{\prime}]$.
$$
[c] = [c^{\prime}] \Rightarrow c-c^{\prime} = \partial d
$$
$$
\phi_{*}([c]) = \phi_{*}(\partial d + c^{\prime}) = [\phi(\partial d+c^{\prime})] = [\phi(\partial d)+\phi(c^{\prime})] = [\partial\phi(d)+\phi(c^{\prime})] = 0 +[\phi(c^{\prime})] = \phi_{*}([c^{\prime}])
$$
A few remarks are in order.
\begin{itemize}
\item $(\phi_{*})_{p}$ is a homomorphism
\item $\phi: \varepsilon\to\epsilon^{\prime}$ and $\Phi:\varepsilon^{\prime}\to\varepsilon^{\prime\prime}$ then $\Phi\circ\phi:\varepsilon\to\varepsilon^{\prime\prime}$
\item $(\Phi\circ\phi)_{*}=\Phi_{*}\circ\phi_{*}$
\item $\text{Id}:\varepsilon\to\varepsilon $ is a chain map then $(\text{Id}_{*})_{p} = \text{Id}_{H_{p}}$
\end{itemize}
\begin{center}
\textcolor{red}{prove these properties. Third and fourth properties are called functorial properties}
\end{center}
\begin{itemize}
\item \textcolor{red}{proof 1}
\item \textcolor{red}{proof 2}
\item \textcolor{red}{proof 3}
\item \textcolor{red}{proof 4}
\end{itemize}
We can use all this now to prove topological invariance of the singular homology groups.
\begin{definition}
Let $f:X\to Y$ be a continuous function.
We define
$$
(f_{#})_{p}: S_{p}(X)\to S_{p}(Y)
$$
$$
(f_{#})_{p}(\sigma) = f\circ \sigma
$$
\end{definition}
We have that $(f_{#})_{p}$ is a chain map, but I dont prove it. At this point we are just defining new functions in terms of compositions of functions, then defining functions on top of that. The reason that most people do not like algebra.
Let $f:X\to Y$ and $g:Y\to Z$ be continuous. They satisfy the functorial properties:
\begin{itemize}
\item $f_{\#}\circ g_{\#} = (f\circ g)_{\#}$
\item For $\text{Id}_{X}:X\to X$ we have that $(\text{Id}_{X})_{\#} = \text{Id}_{\text{Sp}(X)}$
\end{itemize}
Now we define a final function in terms of the other functions to get our prized topological invariance.
\begin{definition}
Let $(f_{#})_{*}:H_{p}(X)\to H_{p}(Y)$ be a homomorphism. It has functorial properties like the other ones.
\begin{corollary}
If $h:X\to Y$ is a homeomorphism then $(h_{#})_{*}:H_{p}(X)\to H_{p}(Y)$ is an isomorphism for all $p$.
\end{corollary}
\end{definition}













\newpage
\begin{exercise}
Prove the following functorial properties. Let $f:X\to Y$ and $g:Y\to Z$ be continuous functions. Then
\begin{enumerate}
\item $g_{\#}\circ f_{\#} = (g\circ f)_{\#}$
\item For $\text{Id}_{X}:X\to X$ we have that $(\text{Id}_{X})_{\#} = \text{Id}_{\text{Sp}(X)}$
\end{enumerate}
\end{exercise}
\solution{
\begin{enumerate}
\item Let $\sigma_{i}$ be  a generator of $S_{p}(X)$. Thus we have
$$
g_{\#}\circ f_{\#}(\sigma_{i}) = g_{\#}(f_{\#}(\sigma_{i})) = g_{\#}(f\circ\sigma_{i}) = g\circ f\circ \sigma_{i} =  (g\circ f)_{\#}(\sigma_{i})
$$
\item  Let $\sigma_{i}$ be  a generator of $S_{p}(X)$. Thus we have
$$
(\text{Id}_{X})_{\#}(\sigma_{i}) = \text{Id}_{X} \circ \sigma_{i} = \sigma_{i}
$$
so $(\text{Id}_{X})_{\#} = \text{Id}_{S_{p}(X)}$
\end{enumerate}
}
\lecture{8}{Reduced Homology, Properties of Singular Homology, Relative Homology and Homological Algebra Part II}
Consider the chain complex
$$
...\rightarrow S_{2}(X)\rightarrow S_{1}(X)\rightarrow S_{0}(X)\rightarrow 0 \rightarrow 0 ...
$$
\begin{definition}
We define $\epsilon:S_{0}(X)\to\mathbb{Z}$ by
$$
\epsilon(\sigma) = 1\ \forall \ \text{0 simplices}\ \sigma
$$
We can prove $\epsilon\circ\sigma = 0$. Then we get the augmented chain complex 
$$
...\rightarrow S_{2}(X)\rightarrow S_{1}(X)\rightarrow S_{0}(X)\xrightarrow{\epsilon}
 \mathbb{Z} \rightarrow 0 ...
$$
With this chain complex we define reduced homology groups
$$
\tilde{H}_{0}(X) = \frac{\ker{\epsilon}}{\partial_{1}S_{1}}
$$
For the rest of the $p$ they are equivalent to rest of homology groups.
\end{definition}
It seems pedantic to create this definition but it will help us in our calculations. It also gets rid of a factor of $\mathbb{Z}$ so $\tilde{H}_{0}(point) = 0$.  Now we move on to properties of the singular homology group.
\begin{proposition}
Let $X = \cup_{\alpha}X_{\alpha}$ be a topological space where each $X_{\alpha}$ are the path connected components. Then 
$$
H_{p}(X)\cong \oplus_{\alpha}H_{p}(X_{\alpha})
$$
\end{proposition}
This is a nice property and I don't prove it. We will use it in a later proof though.
\begin{proposition}
Let $X$ be a topological space. Then $H_{0}(X)$ is free abelian and if $\{\sigma_{\alpha}}\}$ is a family of 0 simplices such that for $\forall$ path connected components $X_{\alpha}$ there existes exactly one $\sigma_{\alpha}$ such that $\sigma_{\alpha}(\Delta_{0})\in X_{\alpha}$ then $\{\sigma_{\alpha}\}$ is a basis for $H_{0}(X)$.
\end{proposition}
I don't prove this one either but we use this to make another proposition.
\begin{proposition}
Let $X$, $X_{\alpha}$ and $\{\sigma_{\alpha}\}$ be as in the previous proposition. Then $\tilde{H}_{0}(X)$ is a free abelian group. If we fix $\alpha_{0}$ then $\{[\sigma_{\alpha}-\sigma_{0}]|\alpha\neq\alpha_{0}\}$ is a basis of $\tilde{H}_{0}(X)$.
\end{proposition}
Now we put all these propositions to use.
\begin{theorem}
If $X = \{x_{0}\}$ as single point then $X$ is acyclic. $\tilde{H}_{p}(X) = 0$ for all $p$
\end{theorem}
This makes sense because the $p^{th}$ homology group measures the number of $p$ dimensional holes of a space. Lets see how easy or difficult it is to get this trivial result.
\begin{proof}
There exists a unique p-simplex (for each p)
$$
\sigma_{p}:\Delta_{p}\to\{x_{0}\}
$$
which is clearly a constant function. Let us consider the chain map
$$
...\xrightarrow{\partial_{2k+1}} S_{2k+1}(X)\xrightarrow{\partial_{2k+1}} S_{2k}(X)\xrightarrow{\partial_{2k}} S_{2k-1}(X)\xrightarrow{\partial_{2k-1}}S_{2k-2}(X)\xrightarrow{\partial_{2k-2}}...
$$
Given that there is only one $\sigma_{p}$ for each $p$ we have that this chain becomes
$$
...\xrightarrow{\partial_{2k+2}} \mathbb{Z}\xrightarrow{\partial_{2k+1}} \mathbb{Z}\xrightarrow{\partial_{2k}} \mathbb{Z}\xrightarrow{\partial_{2k-1}}\mathbb{Z}\xrightarrow{\partial_{2k-2}}...
$$
If we apply the boundary operator to any $\sigma_{p}$ we have
$$
\partial_{p}(\sigma_{p})=\sum_{i=0}^{p}(-1)^{i}\sigma_{p}\circ \varphi_{i} = (\sum_{i}^{p}(-1)^{i})\sigma_{p-1}
$$
which is just adding a constant map over and over. We have
$$
\partial_{p}(\sigma_{p}) = 
\begin{cases}
0,\quad \text{p is odd}\\
\sigma_{p-1},\  \text{p is even}
\end{cases}
$$
So the chain complex is now
$$
...\xrightarrow{Id} \mathbb{Z}\xrightarrow{0} \mathbb{Z}\xrightarrow{Id}\mathbb{Z}\xrightarrow{0}\mathbb{Z}\xrightarrow{Id}...
$$
Let us look at the Homology groups $H_{p}(X)=\ker{\partial_{p}/\Im{\partial_{p+1}}$. We have that $\ker{Id} = 0$, $\Im{Id}=\mathbb{Z}$, $ \ker{0} = \mathbb{Z}$ and $\Im{0} = 0$. This gives us that for p odd we have
$$
\tilde{H}_{p}(X) = \mathbb{Z}/\mathbb{Z} = 0
$$
and for p even we have
$$
\tilde{H}_{p}(X) = 0/0= 0
$$. 
Because $X$ is connected $\tilde{H}_{0}(X) = 0$.
Thus it is acyclic.
\end{proof}
Alot of work for a trivial point. We will move towards doing calculations with exact sequences, which are a better computational tool. Before we do that we define relative homology but I skip over most of the details.\begin{definition}
	Let $A \subseteq X$.  
	The group of relative $p$-chains is defined as the quotient
	\[
	S_p(X,A) = S_p(X) / S_p(A),
	\]
	where $S_p(X)$ and $S_p(A)$ are the singular $p$-chain groups of $X$ and $A$.
	
	The boundary map is induced from the usual one:
	\[
	\partial_p[\sigma] = [\partial_p\sigma],
	\]
	and satisfies $\partial_p^2 = 0$.
	
	The \textit{relative homology groups} of the pair $(X,A)$ are then
	\[
	H_p(X,A) = \ker(\partial_p) / \operatorname{im}(\partial_{p+1}),
	\]
	that is, the homology of the quotient chain complex $S_*(X,A)$.
\end{definition}
Most results carry over from regular homology. Before we get to the main computational tool of exact sequences we return to Homological Algebra to build up some results we will need.
\begin{definition}
Consider a  sequence of abelian groups and homomorphisms
$$
...\xrightarrow{\phi_{i+2}}A_{i+1}\xrightarrow{\phi_{i+1}}A_{i}\xrightarrow{\phi_{i}}...
$$	
The sequence is \textit{exact} if and only if $\Im{\phi_{i+1}}=\ker{\phi_{i}}$
An \textit{exact sequence} is a chain complex with trivial homology. We have a specific type of exact sequence called a \textit{short exact sequence} with the following form
$$
0\rightarrow A\xrightarrow{\Phi}B\xrightarrow{\Psi}C\rightarrow 0
$$
A sequence being short exact is equivalent to the following
\begin{itemize}
\item $\Phi$ is injective
\item $\ker{\Psi} = \Im{\Phi}$
\item $\Psi$ is surjective
\end{itemize}
\end{definition}
We can now specialize this definition to chain complexes.
\begin{definition}[Short exact sequence of chain complexes]
	A sequence of chain complexes
	\[
	0 \longrightarrow (A_{\ast},\partial_{A})
	\xrightarrow{\;\Phi_{\ast}\;}
	(B_{\ast},\partial_{B})
	\xrightarrow{\;\Psi_{\ast}\;}
	(C_{\ast},\partial_{C})
	\longrightarrow 0
	\]
	is \emph{short exact} if for every $p \in \mathbb{Z}$ the sequence
	\[
	0 \longrightarrow A_{p}
	\xrightarrow{\;\Phi_{p}\;}
	B_{p}
	\xrightarrow{\;\Psi_{p}\;}
	C_{p}
	\longrightarrow 0
	\]
	is exact, and the boundary maps satisfy
	\[
	\partial_{B}\Phi_{p} = \Phi_{p-1}\partial_{A},
	\qquad
	\partial_{C}\Psi_{p} = \Psi_{p-1}\partial_{B}.
	\]
	
	\noindent
	This can be visualized as the following ``ladder'' of exact rows:
	\[
	\begin{array}{ccccccccc}
		&  & \vdots &  & \vdots &  & \vdots &  & \\[-0.3em]
		0 &\!\!\longrightarrow\!\!& A_{p+1} &\!\!\xrightarrow{\Phi_{p+1}}\!\!& B_{p+1} &\!\!\xrightarrow{\Psi_{p+1}}\!\!& C_{p+1} &\!\!\longrightarrow\!\!& 0 \\[0.3em]
		&  & \;\;\downarrow\partial_{A} &  & \;\;\downarrow\partial_{B} &  & \;\;\downarrow\partial_{C} &  & \\[0.3em]
		0 &\!\!\longrightarrow\!\!& A_{p} &\!\!\xrightarrow{\Phi_{p}}\!\!& B_{p} &\!\!\xrightarrow{\Psi_{p}}\!\!& C_{p} &\!\!\longrightarrow\!\!& 0 \\[0.3em]
		&  & \downarrow\partial_{A} &  & \downarrow\partial_{B} &  & \downarrow\partial_{C} &  & \\[0.3em]
		&  & \vdots &  & \vdots &  & \vdots &  &
	\end{array}
	\]
\end{definition}


\lecture{9}{Zig-Zag Lemma, more Homological Algebra and Applications to Algebraic Topology}
\begin{lemma}
Let
$$
0\rightarrow \alpha \xrightarrow{\Phi} \beta \xrightarrow{\Psi} \gamma \rightarrow 0
$$
Be a short exact sequence of chain complexes. Then there exists a family of homomorphisms
$$
H_{p}(\gamma)\xrightarrow{\partial_{*}}H_{p-1}(\alpha)
$$
such that the following is a long exact sequence
$$
...\rightarrow H_{p}(\alpha) \xrightarrow{\Phi_{*}} H_{p}(\beta) \xrightarrow{\Psi_{*}} H_{p}(\gamma) \xrightarrow{\partial_{*}} H_{p-1}(\alpha)\xrightarrow{\Phi_{*}}...
$$
where $\Phi_{*}$ and $\Psi_{*}$ are the homomorphisms induced by $\Phi$ and $\Psi$. This is called the or zig-zag lemma.
\end{lemma}
\begin{proof}
We define $\partial_{*}:H_{p}(\gamma)\to H_{p-1}(\delta)$ and consider $[e_{p}]\in H_{p}(\gamma)$ where $e_{p}$ is a cycle. Because $\Psi_{p}$ is surjective there exists $d_{p}$ such that $\Psi_{p}(d_{p}) = e_{p}$. Now consider $\partial_{D}(d_{p})$. By commutativity we have
$$
\Psi_{p-1}(\partial_{D}(d_{p})) = \partial_{E}(\Psi_{p}(d_{p})) = \partial_{E}e_{p} = 0
$$
So $\partial_{D}(d_{p})\in\ker{\Psi_{p-1}} = \Im{\Phi_{p-1}}$. Therefore there exists $c_{p-1}$ such that $\Phi_{p-1}(c_{p-1}) = \partial_{D}(d_{P})$. We define
$$
\partial_{*}([e_{p}]) = [c_{p-1}]
$$
Consider the following diagram for reference
\[
\begin{array}{ccccccccc}
	&  & \vdots &  & \vdots &  & \vdots &  & \\[-0.3em]
	0 &\!\!\longrightarrow\!\!& C_{p} &\!\!\xrightarrow{\boldsymbol{\Phi_{p}}}\!\!& \mathbf{D_{p}} &\!\!\xrightarrow{\boldsymbol{\Psi_{p}}}\!\!& \mathbf{E_{p}} &\!\!\longrightarrow\!\!& 0 \\[0.3em]
	&  & \;\;\downarrow\partial_{C} &  & \;\;\mathbf{\downarrow\partial_{D}} &  & \;\;\mathbf{\downarrow\partial_{E}} &  & \\[0.3em]
	0 &\!\!\longrightarrow\!\!& \mathbf{C_{p-1}} &\!\!\xrightarrow{\boldsymbol{\Phi_{p-1}}}\!\!& \mathbf{D_{p-1}} &\!\!\xrightarrow{\Psi_{p-1}}\!\!& E_{p-1} &\!\!\longrightarrow\!\!& 0 \\[0.3em]
	&  & \;\;\downarrow\partial_{C} &  & \;\;\downarrow\partial_{D} &  & \;\;\downarrow\partial_{E} &  & \\[0.3em]
	0 &\!\!\longrightarrow\!\!& C_{p-2} &\!\!\xrightarrow{\Phi_{p-2}}\!\!& D_{p-2} &\!\!\xrightarrow{\Psi_{p-2}}\!\!& E_{p-2} &\!\!\longrightarrow\!\!& 0 \\[0.3em]
	&  & \vdots &  & \vdots &  & \vdots &  &
\end{array}
\]
In the diagram, the connecting morphism is visualized as a path that moves 
\emph{left, down, and left} across the rows and columns:
\[
E_p 
\;\xleftarrow{\Psi_p}\;
D_p 
\;\xrightarrow{\partial_D}\;
D_{p-1} 
\;\xleftarrow{\Phi_{p-1}}\;
C_{p-1}.
\]
Starting with a cycle in the top right ($E_p$), one moves left to $D_p$ using the surjectivity of 
$\Psi_p$, then downward via the boundary $\partial_D$, and finally left again to $C_{p-1}$ using 
the exactness of the lower row. 
This ``zig--zag'' path through the commutative diagram visually represents the construction of the 
connecting homomorphism $\partial_* : H_p(E)\to H_{p-1}(C)$. 
We still need to prove $\partial_{*}([e_{p}]) = [c_{p-1}]$ is well defined but I skip that part of the proof. Its a few pages of work and not interesting.
\end{proof}
Now we consider the naturality of the long exact sequence in homology.
\begin{theorem}[Naturality of the Long Exact Sequence in Homology]
Consider the following commutative diagram
\[
\begin{array}{ccccccccc}
	0 & \longrightarrow & A & \xrightarrow{\Phi} & B & \xrightarrow{\Psi} & C & \longrightarrow & 0 \\[0.4em]
	&                 & \downarrow \alpha     & \circlearrowright     & \downarrow \beta  & \circlearrowright      & \downarrow \gamma      &   \\[0.2em]
	0 & \longrightarrow & D & \xrightarrow{\Phi'}& E & \xrightarrow{\Psi'}& F & \longrightarrow & 0
\end{array}
\]
where the horizontal rows are short exact sequences of chain complexes and $\alpha,\beta,\gamma$ are chain maps. Then the following ladder of homology groups commutes.
\[
\begin{array}{ccccccccccccc}
	\cdots & \longrightarrow &
	H_p(A) & \xrightarrow{\Phi_*} &
	H_p(B) & \xrightarrow{\Psi_*} &
	H_p(C) & \xrightarrow{\partial_*} &
	H_{p-1}(A) & \xrightarrow{\Phi_*} &
	H_{p-1}(B) & \longrightarrow & \cdots \\[0.6em]
	& & \downarrow \alpha_* & \circlearrowleft &
	\downarrow \beta_* & \circlearrowleft &
	\downarrow \gamma_* & \circlearrowleft &
	\downarrow \alpha_* & \circlearrowleft &
	\downarrow \beta_* &  & \\[0.4em]
	\cdots & \longrightarrow &
	H_p(D) & \xrightarrow{\Phi_*'} &
	H_p(E) & \xrightarrow{\Psi_*'} &
	H_p(F) & \xrightarrow{\partial_*'} &
	H_{p-1}(D) & \xrightarrow{\Phi_*'} &
	H_{p-1}(E) & \longrightarrow & \cdots
\end{array}
\]
\end{theorem}
We can now apply this to algebraic topology.
\begin{theorem}
Let $X$ be a topological space and $A\subseteq X$. Then there exists a homomorphism $\partial_{*}:H_{p}(X,A)\to H_{p-1}(A)$ such that
$$
...\rightarrow H_{p}(A)\xrightarrow{i_{*}}H_{p}(X)\xrightarrow{\pi_{*}}H_{p}(X,A)\xrightarrow{\partial_{*}}H_{p-1}(A)\rightarrow ...
$$
is a long exact sequence where $i_{*}$ is induced by $i:A\to X$. The map $\pi_{*}$ is induced by $$\pi_{\#}:S_{p}(X)\rightarrow\frac{S_{p}(X)}{S_{p}(A)} = S_{p}(X,A)$$
$\pi_{*}$ is a chain map. Moreover if $A\neq \emptyset$ this holds for the reduced homology.
$$
...\rightarrow \tilde{H}_{p}(A)\rightarrow \tilde{H}_{p}(X)\rightarrow H_{p}(X,A)\rightarrow \tilde{H}_{p-1}(A)\rightarrow\tilde{H}_{p-1}(X)\rightarrow ...
$$
($\tilde{H}_{p}(X,A)$ does not exist). If $f(X,A)\to f(Y,B)$ is a continuous map then $f_{*}$ induces a chain map between the long exact sequence in homology
\[
\begin{array}{ccccccccccccc}
	\cdots & \longrightarrow &
	H_p(A) & \rightarrow &
	H_p(X) & \rightarrow &
	H_p(X,A) & \rightarrow&
	H_{p-1}(A) & \longrightarrow & \cdots \\[0.6em]
	& & \downarrow (f|_{A})_{*} & \circlearrowleft &
	\downarrow f_{*} & \circlearrowleft &
	\downarrow f_{*} & \circlearrowleft &
	\downarrow (f|_{A})_{*}  & & \\[0.4em]
	\cdots & \longrightarrow &
	H_p(B) & \rightarrow &
	H_p(Y) & \rightarrow &
	H_p(Y,B) & \rightarrow&
	H_{p-1}(B) & \longrightarrow & \cdots
\end{array}
\]
\end{theorem}
\noindent
The groups $H_{p}(A)$, $H_{p}(X)$, and $H_{p}(X,A)$ measure related but distinct
topological information. The absolute group $H_{p}(A)$ records $p$-dimensional
holes contained entirely in $A$, while $H_{p}(X)$ records those of the whole
space $X$. The relative group $H_{p}(X,A)$ measures the new $p$-dimensional
features of $X$ that disappear when restricted to $A$, that is, the topology of
$X$ \emph{relative} to $A$. The long exact sequence
\[
\cdots \rightarrow H_{p}(A) \xrightarrow{i_*} H_{p}(X)
\xrightarrow{\pi_*} H_{p}(X,A) \xrightarrow{\partial_*} H_{p-1}(A)
\rightarrow \cdots
\]
links these groups together and allows one to compute the homology of a pair
from the homology of its parts. If $f:(X,A)\to(Y,B)$ is a continuous map, the
induced maps on homology make the entire sequence commute, expressing the
\emph{naturality} of the long exact sequence. This could be used, for example, when studying the circle as a subspace of the disc.

\begin{exercise}
	Prove the Snake Lemma.
\end{exercise}
\solution{The Snake Lemma is the more general group theoretic version of zig-zag lemma. ChatGPT gave me this proof, I didn't want to do it. It is exceptionally brutal and purely algebraic.  Let
	\[
	\begin{array}{ccccccccc}
		0 &\longrightarrow& A &\xrightarrow{\ \Phi\ }& B &\xrightarrow{\ \Psi\ }& C &\longrightarrow& 0 \\
		&                 & \downarrow \alpha & & \downarrow \beta & & \downarrow \gamma & & \\
		0 &\longrightarrow& D &\xrightarrow{\ \Phi'\ }& E &\xrightarrow{\ \Psi'\ }& F &\longrightarrow& 0
	\end{array}
	\]
	be a commutative diagram of Abelian groups with exact rows. Then there is a natural exact sequence
	\[
	0 \longrightarrow \ker\alpha \xrightarrow{\ \overline{\Phi}\ } \ker\beta
	\xrightarrow{\ \overline{\Psi}\ } \ker\gamma
	\xrightarrow{\ \delta\ } \operatorname{coker}\alpha
	\xrightarrow{\ \overline{\Phi'}\ } \operatorname{coker}\beta
	\xrightarrow{\ \overline{\Psi'}\ } \operatorname{coker}\gamma
	\longrightarrow 0,
	\]
	where $\overline{\Phi},\overline{\Psi}$ are restrictions, $\overline{\Phi'},\overline{\Psi'}$ are maps induced on cokernels, and $\delta$ is the \emph{connecting homomorphism} defined below.
\newline

First we start with  $\delta:\ker\gamma\to \operatorname{coker}\alpha$.}
	Let $c\in \ker\gamma\subseteq C$. Since $\Psi$ is surjective, pick $b\in B$ with $\Psi(b)=c$.
	Commutativity gives $\Psi'(\beta(b))=\gamma(\Psi(b))=\gamma(c)=0$, hence $\beta(b)\in\ker\Psi'=\operatorname{im}\Phi'$ (by exactness). So choose $d\in D$ with $\Phi'(d)=\beta(b)$. Define
	\[
	\delta(c):=[d]\ \in\ \operatorname{coker}\alpha=D/\operatorname{im}\alpha.
	\]
	
Now we check well-definedness.
	If $b$ and $b'$ both map to $c$, then $b-b'\in\ker\Psi=\operatorname{im}\Phi$, say $b-b'=\Phi(a)$.
	Applying $\beta$ and using commutativity:
	\[
	\beta(b)-\beta(b')=\beta(\Phi(a))=\Phi'(\alpha(a)),
	\]
	so the corresponding $d,d'$ differ by an element of $\operatorname{im}\alpha$, hence $[d]=[d']$ in $D/\operatorname{im}\alpha$. Thus $\delta$ is well-defined and is a homomorphism.
	\newline
	
	Now check exactness of the following sequence
	 $\ker\alpha\to\ker\beta\to\ker\gamma$
	The map $\overline{\Phi}:\ker\alpha\to\ker\beta$ is $x\mapsto \Phi(x)$; it lands in $\ker\beta$
	since $\beta(\Phi(x))=\Phi'(\alpha(x))=0$. Injectivity: if $\Phi(x)=0$ then $x=0$ by exactness.
	Next, $\overline{\Psi}:\ker\beta\to\ker\gamma$ is $y\mapsto \Psi(y)$; it lands in $\ker\gamma$
	since $\gamma(\Psi(y))=\Psi'(\beta(y))=0$. If $y=\Phi(x)$ with $x\in\ker\alpha$, then
	$\overline{\Psi}(y)=\Psi(\Phi(x))=0$, so $\operatorname{im}\overline{\Phi}\subseteq\ker\overline{\Psi}$.
	\newline
	
	Conversely, if $y\in\ker\beta$ and $\Psi(y)=0$, then $y\in\ker\Psi=\operatorname{im}\Phi$, say $y=\Phi(x)$,
	and $\beta(y)=0$ implies $\Phi'(\alpha(x))=0$, hence $\alpha(x)=0$ by injectivity of $\Phi'$.
	So $x\in\ker\alpha$ and $y\in\operatorname{im}\overline{\Phi}$. Thus
	\[
	\operatorname{im}\overline{\Phi}=\ker\overline{\Psi}.
	\]
	
Now we check exactness of the middle part of the sequence, $\ker\gamma\ \xrightarrow{\ \delta\ }\ \operatorname{coker}\alpha\ \xrightarrow{\ \overline{\Phi'}\ }\ \operatorname{coker}\beta$.}
	First, $\overline{\Phi'}\circ \delta=0$: with notation above, $\overline{\Phi'}([d])=[\Phi'(d)]=[\beta(b)]$
	in $\operatorname{coker}\beta$, which is $0$ by definition, so $\operatorname{im}\delta\subseteq\ker\overline{\Phi'}$.
	\newline
	
	Conversely, suppose $[d]\in\ker\overline{\Phi'}$, i.e.\ $[\Phi'(d)]=0$ in $E/\operatorname{im}\beta$.
	Then $\Phi'(d)=\beta(b)$ for some $b\in B$. Set $c=\Psi(b)\in C$. Commutativity gives
	$\gamma(c)=\Psi'(\beta(b))=\Psi'(\Phi'(d))=0$, so $c\in\ker\gamma$. By construction,
	$\delta(c)=[d]$. Hence $\ker\overline{\Phi'}\subseteq\operatorname{im}\delta$ and thus
	\[
	\operatorname{im}\delta=\ker\overline{\Phi'}.
	\]
	
Finally, we check exactness of the last part of the sequence, $\operatorname{coker}\alpha\ \xrightarrow{\ \overline{\Phi'}\ }\ \operatorname{coker}\beta\ \xrightarrow{\ \overline{\Psi'}\ }\ \operatorname{coker}\gamma$.}
	For $[d]\in \operatorname{coker}\alpha$, $\overline{\Psi'}(\overline{\Phi'}([d]))=[\Psi'(\Phi'(d))]
	=[\gamma(\Psi(\_))]=0$ by commutativity, so $\operatorname{im}\overline{\Phi'}\subseteq\ker\overline{\Psi'}$.
	\newline
	
	Conversely, if $[e]\in\ker\overline{\Psi'}$ (so $[\Psi'(e)]=0$ in $F/\operatorname{im}\gamma$), then
	$\Psi'(e)=\gamma(c)$ for some $c\in C$. Choose $b\in B$ with $\Psi(b)=c$. Then
	\[
	\Psi'(e-\beta(b))=\Psi'(e)-\Psi'(\beta(b))=\gamma(c)-\gamma(\Psi(b))=0,
	\]
	so $e-\beta(b)\in\ker\Psi'=\operatorname{im}\Phi'$, say $e-\beta(b)=\Phi'(d)$.
	Thus $[e]=[\Phi'(d)]=\overline{\Phi'}([d])$. Hence
	\[
	\operatorname{im}\overline{\Phi'}=\ker\overline{\Psi'}.
	\]
	
	Finally, consider the ends of the sequence.
	$\overline{\Phi}$ is injective by exactness of the top row at $A$; $\overline{\Psi'}$ is
	surjective by surjectivity of $\Psi'$. This yields the full exact sequence as stated.
}
}
\begin{exercise}
	Let $X$ be a subset of $\mathbb{R}^n$ containing $w$ such that for all $x \in X$ 
	the segment connecting $x$ and $w$ is contained in $X$ 
	(the set $X$ is said to be \emph{stair-convex} with center $w$). 
	Prove that $X$ is acyclic with respect to the singular homology.
	
	\medskip
	\noindent
	\textit{Hint:} recall that $z \in \Delta_{p+1}$ can be expressed as 
	$z = t e_{p+1} + (1-t)z'$ where $z' \in \Delta_p$; 
	given a singular $p$-simplex $\sigma : \Delta_p \to X$, 
	one can construct a singular $(p+1)$-simplex 
	$[w, \sigma] : \Delta_{p+1} \to X$ where 
	$z = t e_{p+1} + (1-t)z' \in \Delta_{p+1}$ is mapped to 
	$t w + (1-t)\sigma(z')$.
\end{exercise}
\solution{
\begin{center}
\textcolor{red}{insert solution here}
\end{center}
}
}
\lecture{10}{Homotopy Review}
\begin{definition}
Let $X$ and $Y$ be topological spaces and let $f, g : X \to Y$ be continuous maps.  
We say that $f$ and $g$ are \emph{homotopic}, written $f \simeq g$, if there exists a continuous map
\[
H : X \times [0,1] \to Y
\]
such that
\[
H(x,0) = f(x) \quad \text{and} \quad H(x,1) = g(x) \quad \text{for all } x \in X.
\]
The map $H$ is called a \emph{homotopy} between $f$ and $g$.
\end{definition}
\begin{figure}[H]
	\centering
	\includegraphics[width=0.3\textwidth]{figures/homotopy.png}
	\label{fig:my_diagram}
\end{figure}
In other words, they map into the same space and can be deformed in to eachother without ever leaving the space. This homotopy which defines the fundamental group also helps us in homology.
\begin{lemma}[Homotopy Lemma]
	Let $f, g : X \to Y$ be continuous maps between topological spaces. 
	If $f$ and $g$ are homotopic, then they induce the same homomorphism on homology:
	\[
	f_* = g_* : H_n(X) \to H_n(Y) \quad \text{for all } n \ge 0.
	\]
\end{lemma}
This holds for the reduced and unreduced homology groups.
\begin{corollary}[Homotopy Invariance of Homology]
	If two topological spaces $X$ and $Y$ are homotopy equivalent, 
	then they have isomorphic homology groups:
	\[
	H_p(X) \cong H_p(Y) \quad \text{for all } p \ge 0.
	\]
\end{corollary}
The simplest spaces are described as contractible.
\begin{definition}[Contractible Space]
	A topological space $X$ is said to be \emph{contractible} if the identity map
	\[
	\mathrm{id}_X : X \to X
	\]
	is homotopic to a constant map $c_{x_0} : X \to X$, where $c_{x_0}(x) = x_0$ for some fixed point $x_0 \in X$.
	\end{definition}
	With the following definition we can connext contractible spaces to singular homology.
\begin{remark}
	A space $X$ is contractible if and only if the identity map 
	$\mathrm{id}_X : X \to X$ is homotopic to a constant map 
	$c_{x_0} : X \to X$, where $c_{x_0}(x) = x_0$ for some $x_0 \in X$.
\end{remark}
We conclude with the following corollary that connects homotopy and singular homology.
\begin{corollary}
A contractible space is acyclic.
\end{corollary}
\lecture{11}{Star Convexity, Excision Theorem and Homology of Spheres}
\begin{definition}
$X\subseteq\mathbb{R}^{n}$ is called a star convex set if $\exists x_{0}\in X$ such that every $x$ has a straight line that connects to it that is also contained in $X$
\end{definition}
\begin{remark}
Every convex set in $\mathbb{R}^{n}$ is star convex, but not every star convex set is convex.
\begin{figure}[H]
	\centering
	\includegraphics[width=0.3\textwidth]{figures/convex.png}
	\label{fig:my_diagram}
\end{figure}
\end{remark}
We are now ready to discuss the excision theorem.
\begin{theorem}
Let $X$ be a topological space and $A \subseteq X$. If $U\subseteq X$ and $\overline{U}\subseteq\text{Int}A$ then the inclusion map
$$
j: (X,A)\to(X\setminus U, A\setminus U)
$$
induces and isomorphism in homology 
$$
j_{*}:H_{p}(X\setminus U, A\setminus U)\to H_{p}(X,A)\ \forall p
$$
\end{theorem}
The excision theorem means that if a subset $U$ lies entirely inside the interior of $A$,
then removing $U$ from both $X$ and $A$ does not change the relative homology.
In other words, the homology groups $H_p(X,A)$ depend only on how $A$ meets the boundary
of $X$, not on what happens deep inside $A$.
Thus, any region $U$ completely contained in $\mathrm{Int}(A)$ can be ``excised''
without affecting the homology of the pair.
\begin{definition}
Let $X$ be a topological space and $\mu=\{U_{j}\}$ be a family of subsets of $X$ such that $X = \cup_{j}\text{Int}U_{j}$. A singular $\sigma:\Delta_{p}\to X$ is called $\mu-$small if there exists a $j_{0}$ such that $\sigma(\Delta_{p})\subseteq U_{j_{0}}$
We define $S_{p}^{\mu}(X)$ as
$$
S_{p}^{\mu}(X) = \langle\sigma \text{is a p simplex}|\sigma\text{ is $\mu-$small}\rangle
$$
. $S_{p}^{\mu}(X)\subseteq S_{p}(X)$ is the group of $\mu-$small p-chains.
\end{definition}
The idea of $\mu$-small simplices is to control where each simplex lies with respect to
an open cover $\mu=\{U_j\}$ of $X$.
A simplex is called $\mu$-small if its entire image is contained in one element of the
cover.
Intuitively, this means the simplex is ``small enough'' not to cross between different
sets of the cover.
Such $\mu$-small chains form a subcomplex of the singular chain complex and are useful
in the proof of the excision theorem.
\begin{remark}
	The boundary operator $\partial_p : S_p(X) \to S_{p-1}(X)$
	preserves $\mu$-smallness.
	Indeed, if $\sigma$ is $\mu$-small with
	$\sigma(\Delta_p) \subseteq U_{j_0}$,
	then each face of $\sigma$ also lies in $U_{j_0}$,
	so $\partial_p\sigma$ is a $\mu$-small $(p-1)$-chain.
	Hence, the subgroups $S_p^{\mu}(X)$ form a subchain complex of the singular chain complex:
	\[
	\cdots
	\;\xrightarrow{\partial_{p+1}}\;
	S_p^{\mu}(X)
	\;\xrightarrow{\partial_p}\;
	S_{p-1}^{\mu}(X)
	\;\xrightarrow{\partial_{p-1}}\;
	\cdots
	\;\xrightarrow{\partial_1}\;
	S_0^{\mu}(X)
	\;\to\; 0.
	\]
\end{remark}
We can define $H_{p}^{\mu}(X)$ as the homology groups of this chain complex. 
\begin{theorem}
	Let $X$ be a topological space and $\mu=\{U_j\}$ an open cover of $X$
	such that $X = \bigcup_j \mathrm{Int}\,U_j$.
	Then the inclusion of chain complexes
	\[
	i : S_*^{\mu}(X) \hookrightarrow S_*(X)
	\]
	induces an isomorphism in homology:
	\[
	i_* : H_p(S_*^{\mu}(X)) \xrightarrow{\;\cong\;} H_p(X)
	\quad \text{for all } p \ge 0.
	\]
\end{theorem}
We are now ready to prove the excision theorem, which we will use to calculate our first not trivial homology of the spheres $S^{n}$.
\begin{proof}
Let $X$ be a topological space, $A \subseteq X$, and $U \subseteq X$ an open subset
such that $\overline{U} \subseteq \mathrm{Int}(A)$.

We define the family
\[
\mu = \{A,\, X \setminus \overline{U}\}.
\]
We claim that $\mu$ satisfies the condition
\[
X = \mathrm{Int}(A) \,\cup\, \mathrm{Int}(X \setminus \overline{U}),
\]
which allows us to apply the $\mu$-small theorem.

Indeed, since $\overline{U} \subseteq \mathrm{Int}(A)$, for any $x \in X$ we have two possibilities:

\begin{itemize}
	\item If $x \in \overline{U}$, then $x \in \mathrm{Int}(A)$.
	\item If $x \notin \overline{U}$, then $x \in X \setminus \overline{U}
	\subseteq \mathrm{Int}(X \setminus U)
	\subseteq \mathrm{Int}(X \setminus \overline{U})$.
\end{itemize}
Thus every $x \in X$ belongs to $\mathrm{Int}(A) \cup \mathrm{Int}(X \setminus \overline{U})$,
and therefore
\[
X = \mathrm{Int}(A) \cup \mathrm{Int}(X \setminus \overline{U}).
\]
This verifies that $\mu$ is an open cover of $X$ in the sense required by
the $\mu$-small simplices theorem.

Consequently, by the $\mu$-small theorem, the inclusion map
$$
i : S^{\mu}(X,A) \hookrightarrow S(X,A)
$$
induces an isomorphism in homology. Recall the definition of relative homology
$$
S^{\mu}(X,A)= \frac{S^{\mu}(X)}{S^{\mu}(A)}
$$. Now the claim is that this inclusion map is an isomorphsim
$$
\frac{S_{p}(X\setminus U)}{S_{p}(A\setminus U)}\to\frac{S_{p}(X)}{S_{p}(A)}$$ is an isomporphism.
To prove this we first define the auxilary map
\[
\phi : S_p(X \setminus U) \longrightarrow \frac{S_p^{\mu}(X)}{S_p^{\mu}(A)}
\]
as the composition of the inclusion and projection maps:
\[
S_p(X \setminus U)
\;\xrightarrow{\;\text{inclusion}\;}\;
S_p^{\mu}(X)
\;\xrightarrow{\;\text{projection}\;}\;
\frac{S_p^{\mu}(X)}{S_p^{\mu}(A)}.
\]
That is,
\[
\phi = \text{projection} \circ \text{inclusion}.
\]
Now let
\[
c_p + S_p^{\mu}(A) \;\in\; \frac{S_p^{\mu}(X)}{S_p^{\mu}(A)}.
\]
Then $c_p$ is a $\mu$-small chain. Hence $c_p$ is a sum of simplices
whose images are contained either in $A$ or in $X \setminus U$.

If $\sigma_i : \Delta_p \to X$ is one such simplex and
$\sigma_i(\Delta_p) \subseteq A$, then
\[
\sigma_i + S_p^{\mu}(A) = S_p^{\mu}(A)
\quad\text{in}\quad \frac{S_p^{\mu}(X)}{S_p^{\mu}(A)}.
\]

Therefore, there exists a chain
\[
c_p' = \sum \text{(simplices with image contained in $X \setminus U$)}
\]
such that
\[
c_p' + S_p^{\mu}(A) = c_p + S_p^{\mu}(A).
\]

We note that $c_p' \in S_p(X \setminus U)$, and
\[
\phi(c_p') = c_p' + S_p^{\mu}(A)
= c_p + S_p^{\mu}(A).
\]
Hence $\phi$ is surjective.
We now claim that
\[
\ker \phi = S_p(A \setminus U).
\]

\textbf{($\subseteq$ direction).}
Let $c_p \in S_p(A \setminus U)$.
Then by definition,
\[
\phi(c_p)
= c_p + S_p^{\mu}(A)
= S_p^{\mu}(A)
= 0
\quad \text{in } \frac{S_p^{\mu}(X)}{S_p^{\mu}(A)}.
\]
Hence $c_p \in \ker \phi$.

\medskip

\textbf{($\supseteq$ direction).}
Let $c_p \in S_p(X \setminus U)$ such that
\[
\phi(c_p) = c_p + S_p^{\mu}(A) = 0.
\]
This means $c_p \in S_p^{\mu}(A)$.

Therefore, every simplex of $c_p$ has image contained in $A$.
Since $c_p \in S_p(X \setminus U)$ as well, the images of its simplices
are disjoint from $U$.
Combining these two facts, we conclude that
\[
c_p \in S_p(A \setminus U).
\]

\medskip

Thus both inclusions hold, and therefore
\[
\ker \phi = S_p(A \setminus U).
\]
By the First Isomorphism Theorem, since $\ker\phi=S_p(A\setminus U)$ and $\phi$ is surjective, we have
$$
\frac{S_{p}(X\setminus U)}{S_{p}(A\setminus U)} \cong \frac{S_{p}(X)}{S_{p}(A)} = S_{p}(X,A)
$$
Passing to homology, this chain-level isomorphism induces
\[
\overline{\phi}_* : H_p(X\setminus U,\, A\setminus U)
\;\xrightarrow{\;\cong\;}\;
H_p^{\mu}(X,A).
\]
Since the inclusion $i : S_*^{\mu}(X,A) \hookrightarrow S_*(X,A)$
induces an isomorphism in homology by the $\mu$-small theorem,
we obtain that
\[
j_* = i_* \circ \phi_* :
H_p(X\setminus U,\, A\setminus U)
\;\xrightarrow{\;\cong\;}\;
H_p(X,A)
\]
for all $p$.
\end{proof}
So if we are studying the relative homology of a space, we can remove a piece of the contained entirely withint the subset without affecting the homology groups. This will allow to get homology groups for the spheres.
\begin{theorem}
$$
\tilde{H}_{i}(S^{n}) = 0 \text{ if } i\neq n
$$
When $i=n$
$$
\tilde{H}_{i}(S^{n}) = \mathbb{Z}
$$
\end{theorem}
\begin{proof}
We proceed by induction. For $n=0$, we have the space $S^{0}=\{-1,1\}$. This is a disjoint space so 
$$
H_{0}(S^{0}) \cong H_{0}(-1)\oplus H_{0}(1) \cong \mathbb{Z}\oplus\mathbb{Z} \Rightarrow \tilde{H}_{0}(S^{0})\cong \mathbb{Z}
$$
All the higher homology groups are zero because the points are 0 dimensional. Now we proceed to the inductive step. For $n>0$, we have \[
S^{n}
=\left\{
x=(x_0,x_1,\dots,x_n)\in\mathbb{R}^{n+1}
\;\middle|\;
\sum_{i=0}^{n}x_i^2=1
\right\}.
\]
We also define 
the upper hemisphere:
$$
E_+^{\,n}
=\left\{
(x_0,x_1,\dots,x_n)\in S^{n}
\;\middle|\;
x_n \ge 0
\right\}.
$$
and
thelower hemisphere:
$$
E_-^{\,n}
=\left\{
(x_0,x_1,\dots,x_n)\in S^{n}
\;\middle|\;
x_n \le 0
\right\}.
$$
with their 
common boundary (the ``equator''):
$$
S^{n-1}
=\left\{
(x_0,x_1,\dots,x_n)\in S^{n}
\;\middle|\;
x_n = 0
\right\} = E^{n}_{+}\cap E^{n}_{-}
$$
	\begin{figure}[H]
	\centering
	\includegraphics[width=0.6\textwidth]{figures/SphereExcision.png}
	\label{fig:my_diagram}
\end{figure}
We also define the south pole $q  = \{-1,0,...,0\}$. The next step requires a  couple of definitions which we put here in the middle of this proof.
\begin{definition}[Retraction]
	Let $X$ be a topological space and let $A \subseteq X$ be a subspace.  
	A continuous map 
	\[
	r : X \to A
	\]
	is called a \emph{retraction} if 
	\[
	r|_A = \operatorname{Id}_A.
	\]
	In this case, $A$ is called a \emph{retract} of $X$.
\end{definition}

\begin{remark}
	Let $i : A \hookrightarrow X$ denote the inclusion map.  
	Then $r : X \to A$ is a retraction if and only if 
	\[
	r \circ i = \operatorname{Id}_A.
	\]
\end{remark}

\begin{definition}[Deformation Retraction]
	A retraction $r : X \to A$ is called a \emph{deformation retraction} if the composition
	\[
	i \circ r : X \to X
	\]
	is homotopic to the identity map $\operatorname{Id}_X$, i.e.
	\[
	i \circ r \simeq \operatorname{Id}_X.
	\]
	In this case, we say that $A$ is a \emph{deformation retract} of $X$.
\end{definition}

\begin{remark}
	A deformation retraction is a special kind of homotopy equivalence.
\end{remark}
Our next step is to establish
\[
H_i(E^{n}_{+}, S^{n-1})
\;\xrightarrow[\text{isom.}]{\cong}\;
H_{i-1}(S^{n-1}), 
\quad \forall i.
\]
by finding a long exact sequence in Homology for the pair $(E^{n}_{+},S^{n-1})$.
Since $S^{n-1}$ is a subspace of the closed upper hemisphere $E_+^{\,n}$, 
we can consider the homology of the pair $(E_+^{\,n},S^{n-1})$. 
From the definition of relative homology, there is a long exact sequence connecting 
the homology of the subspace, the total space, and the pair:
\[
\cdots \longrightarrow 
H_i(S^{n-1})
\xrightarrow{i_*}
H_i(E_+^{\,n})
\xrightarrow{j_*}
H_i(E_+^{\,n},S^{n-1})
\xrightarrow{\partial}
H_{i-1}(S^{n-1})
\xrightarrow{i_*}
H_{i-1}(E_+^{\,n})
\longrightarrow \cdots
\]

Because $E_+^{\,n}$ is homeomorphic to a closed $n$-disk, it is contractible and therefore has
\[
H_i(E_+^{\,n}) =
\begin{cases}
	\mathbb{Z}, & i = 0,\\
	0, & i > 0.
\end{cases}
\]
Substituting this into the sequence gives, for all $i \ge 2$,
\[
0 \longrightarrow H_i(E_+^{\,n},S^{n-1})
\xrightarrow{\;\partial\;}
H_{i-1}(S^{n-1})
\longrightarrow 0,
\]
so that
\[
H_i(E_+^{\,n},S^{n-1}) \cong H_{i-1}(S^{n-1}).
\]
The remaining low-degree terms can be checked similarly, confirming that the same relationship holds for all $i$. 
Hence we obtain
\[
H_i(E_+^{\,n}, S^{n-1})
\;\xrightarrow[\text{isom.}]{\cong}\;
H_{i-1}(S^{n-1}), 
\quad \forall i.
\]
Next, we relate the pair $(E^{n}_{+},S^{n-1})$ to the sphere $S^{n}$ itself. 
Consider the inclusion 
\[
i : E_+^{\,n} \hookrightarrow S^n \setminus \{q\}
\]
and define a map
\[
r : S^n \setminus \{q\} \longrightarrow E_+^{\,n}
\]
by projecting each point of the lower hemisphere along its great circle through $q$ to the upper hemisphere. 
\begin{figure}[H]
	\centering
	\includegraphics[width=0.6\textwidth]{figures/retraction.png}
	\label{fig:my_diagram}
\end{figure}
Geometrically, $r$ fixes every point of $E_+^{\,n}$ and sends the lower hemisphere continuously onto it. 
This map satisfies
\[
r \circ i = \operatorname{Id}_{E_+^{\,n}},
\qquad 
i \circ r \simeq \operatorname{Id}_{S^n \setminus \{q\}},
\]
so $r$ is a deformation retraction of $S^n \setminus \{q\}$ onto $E_+^{\,n}$.
Because $r$ restricts to the identity on the common boundary $S^{n-1}$, 
it induces a homotopy equivalence of pairs:
\[
r : (S^n \setminus \{q\}, E_+^{\,n} \setminus \{q\})
\;\simeq\;
(E_+^{\,n}, S^{n-1}).
\]
By the homotopy invariance of homology, this gives
\[
H_i(E_+^{\,n}, S^{n-1})
\;\xrightarrow[\text{isom.}]{r_*}\;
H_i(S^n \setminus \{q\}, E_+^{\,n} \setminus \{q\}),
\quad \forall i.
\]
We also consider the pair $(E_+^{\,n},S^{n-1})$ to $(S^n,E_-^{\,n})$ and  the long exact sequence in reduced homology
\[
\cdots \longrightarrow 
\tilde{H}_i(E_-^{\,n})
\longrightarrow 
\tilde{H}_i(S^n)
\longrightarrow 
H_i(S^n,E_-^{\,n})
\longrightarrow 
\tilde{H}_{i-1}(E_-^{\,n})
\longrightarrow \cdots
\]
Since $E_-^{\,n}$ is also contractible, we have 
$\tilde{H}_i(E_-^{\,n}) = 0$ for all $i$, and the sequence reduces to
\[
\tilde{H}_i(S^n)
\;\xrightarrow[\text{isom.}]{\;\;\;\;\;}
H_i(S^n,E_-^{\,n}),
\quad \forall i.
\]
Next, we apply the Excision Theorem
Taking $X = S^n$, $A = E_-^{\,n}$, and $U$ to be a small neighborhood of the south pole $q$, 
we have $\overline{U} \subset \operatorname{Int}(E_-^{\,n})$, so excision yields
\[
H_i(S^n,E_-^{\,n}) \;\cong\; H_i(S^n \setminus \{q\}, E_+^{\,n} \setminus \{q\}).
\]
Combining this with the previous isomorphism gives
\[
H_i(S^n,E_-^{\,n})
\;\cong\;
H_i(E_+^{\,n},S^{n-1}).
\]
From the inductive step, we already know that
\[
H_i(E_+^{\,n},S^{n-1}) \cong H_{i-1}(S^{n-1}).
\]
Therefore,
\[
\tilde{H}_i(S^n)
\;\cong\;
H_i(S^n,E_-^{\,n})
\;\cong\;
H_i(E_+^{\,n},S^{n-1})
\;\cong\;
H_{i-1}(S^{n-1}),
\]
which completes the induction.  
\[
H_{i-1}(S^{n-1}) =
\begin{cases}
	\mathbb{Z}, & i-1 = n-1,\\
	0, & \text{otherwise}.
\end{cases}
\]
Hence
\[
\tilde{H}_i(S^n) =
\begin{cases}
	\mathbb{Z}, & i = n,\\
	0, & i \neq n.
\end{cases}
\]
\end{proof}
\begin{exercise}
	Let $X$ be a topological space and let $x_0 \in X$.  
	Prove that for all $i \in \mathbb{Z}$ we have
	\[
	\widetilde{H}_i(X) \cong H_i(X, x_0).
	\]
\end{exercise}
\solution{
	Since we aim to find an isomorphism between homology groups, it is enough to construct an
	isomorphism between the corresponding chain complexes. Because reduced homology only differs from ordinary homology in degree $n=0$, we start with the
	simpler case $n>0$. For each $n \ge 0$, the group $S_n(X)$ consists of all finite integer combinations of singular
	$n$-simplices in $X$.  Concretely, a typical element of $S_n(X)$ is
	\[
	c = a_1\sigma_1 + a_2\sigma_2 + \cdots + a_k\sigma_k,
	\]
	where each $\sigma_i : \Delta^n \to X$ is a continuous map (called a singular simplex) and
	$a_i \in \mathbb{Z}$ are integer coefficients.  
	We can think of $c$ as a formal sum of oriented $n$–dimensional pieces drawn in $X$. The boundary operator $\partial_n : S_n(X) \to S_{n-1}(X)$ is defined on each simplex by
	\[
	\partial_n(\sigma)
	= \sum_{i=0}^n (-1)^i\, \sigma|_{[v_0,\dots,\widehat{v_i},\dots,v_n]},
	\]
	that is, the alternating sum of its $(n-1)$–dimensional faces.
	Extending this linearly gives the full chain complex
	\[
	\dots \xrightarrow{\partial_{n+1}} 
	S_n(X)
	\xrightarrow{\partial_n}
	S_{n-1}(X)
	\xrightarrow{\partial_{n-1}}
	\dots
	\xrightarrow{\partial_1}
	S_0(X) \to 0.
	\]
	Now we fix a basepoint $x_0 \in X$.  
	Among all singular simplices $\sigma : \Delta^n \to X$,
	there are the constant ones that send every point of $\Delta^n$
	to $x_0$.  Each such map $\sigma_{x_0}$ is ``entirely collapsed'' inside $x_0$ and has no geometric
	extent.  We denote by $S_n(x_0)$ the set of all integer combinations of these constant simplices:
	\[
	S_n(x_0) = 
	\{\,a_1\sigma_{x_0}^{(1)} + \cdots + a_m\sigma_{x_0}^{(m)} \mid a_i \in \mathbb{Z}\,\}.
	\]
	Since all constant simplices have the same image point $x_0$, this group is essentially
	a copy of $\mathbb{Z}$ sitting inside $S_n(X)$.
	The relative chain group is the quotient
	\[
	S_n(X,x_0) = S_n(X) / S_n(x_0).
	\]
	This means that in $S_n(X,x_0)$ we identify any two chains that differ by an element of
	$S_n(x_0)$.  Equivalently, we declare all chains lying completely inside the point $x_0$ to be zero.
	Formally, for a chain $c \in S_n(X)$, its equivalence class is
	\[
	[c] = c + S_n(x_0)
	= \{\,c + d \mid d \in S_n(x_0)\,\}.
	\]
	Two chains $c_1,c_2 \in S_n(X)$ represent the same element of $S_n(X,x_0)$
	if and only if $c_1 - c_2$ consists entirely of constant simplices at $x_0$.  For $n>0$, we now define
	\[
	\Phi_n : \widetilde{S}_n(X) =S_n(X)
	\longrightarrow S_n(X,x_0)
	\]
	by sending each chain to its equivalence class modulo the constant simplices:
	\[
	\Phi_n(c) = [c] = c + S_n(x_0).
	\]
	Explicitly, on a single simplex $\sigma$,
	\[
	\Phi_n(\sigma) =
	\begin{cases}
		[\sigma], & \text{if $\sigma$ is not constant},\\[4pt]
		0, & \text{if $\sigma(t) = x_0$ for all $t$.}
	\end{cases}
	\]
	Linearly extending this rule to sums of simplices gives $\Phi_n(c) = \sum_i a_i[\sigma_i]$.
	Geometrically, $\Phi_n$ simply ``forgets'' any collapsed simplices sitting at the basepoint. Now we check that $\Phi_{n}$ is an isomorphism. Every element of $S_n(X,x_0)$ is by definition an equivalence class $[c]$ with
	$c \in S_n(X)$.  But $\Phi_n(c) = [c]$ for all $c$, so every class is hit.
	Hence $\Phi_n$ is surjective. Now 
	suppose $\Phi_n(c) = 0$.  
	Then $[c] = 0$ in the quotient, which means $c \in S_n(x_0)$.  
	But $S_n(x_0)$ consists precisely of chains supported entirely at $x_0$, which
	are declared to be zero in both theories for $n>0$.  
	Thus $\Phi_n$ has trivial kernel, and it is injective. So $\Phi$ is an isomorphism. But we need it to comput with the boundary opearator to have a chain complex.
	For any $c \in S_n(X)$ we have
	\[
	\Phi_{n-1}(\partial_n c)
	= [\,\partial_n c\,]
	= \partial^{\mathrm{rel}}_n([c])
	= \partial^{\mathrm{rel}}_n(\Phi_n(c)).
	\]
	Hence $\Phi$ commutes with the boundary operator, and therefore is a chain map.
Now we must adress $n=0$.
	Here, reduced chains differ from ordinary ones:
	\[
	\widetilde{S}_0(X) = \ker(\varepsilon : S_0(X)\to\mathbb{Z}), \qquad
	\varepsilon\!\left(\sum_i a_i[x_i]\right) = \sum_i a_i.
	\]
	This means $\widetilde{S}_0(X)$ consists of integer combinations of points whose total coefficient is
	zero---in other words, formal differences of points in $X$.  We define
	\[
	\Phi_0 : \widetilde{S}_0(X) \longrightarrow S_0(X,x_0)
	\]
	by the same rule $\Phi_0(c) = [c] = c + S_0(x_0)$.
	To see that this is an isomorphism, note that every class $[c]\in S_0(X,x_0)$ has a representative $c' = c - m[x_0]$
	with $\varepsilon(c')=0$, so $\Phi_0$ is surjective.
If $c\in\ker\varepsilon$ and $\Phi_0(c)=0$, then $c\in S_0(x_0)$,
	hence $c=k[x_0]$ for some $k\in\mathbb{Z}$.
	But $\varepsilon(c)=k=0$, so $c=0$.  
	Thus $\Phi_0$ is injective. Therefore $\Phi_0$ is also an isomorphism.
	
Since we have an isomorphism between the chain complexes
	\[
	\Phi_* : \widetilde{S}_*(X) \;\xrightarrow{\;\cong\;}\; S_*(X,x_0),
	\]
	we also have isomorphisms between the corresponding homology groups
	\[
	\widetilde{H}_i(X) \;\cong\; H_i(X,x_0)
	\qquad \text{for all } i \in \mathbb{Z}.
	\]
This means that reduced Intuitively, the reduced homology groups $\widetilde{H}_n(X)$ remove the redundant information
that ordinary homology $H_n(X)$ carries about the mere existence of a point.
In standard homology, each connected component contributes a copy of $\mathbb{Z}$ in degree $0$,
coming from constant simplices that collapse the whole space to a single point.
Reduced homology ``forgets'' this trivial contribution, keeping only the part that reflects the
actual shape of $X$.
Thus both $\widetilde{H}_n(X)$ and $H_n(X,x_0)$ describe the same essential topological
information: they ignore the simplices supported entirely at the basepoint and record only how
$X$ differs from a single point.
}

\begin{exercise}
	Compute the homology groups of the following subspace of $\mathbb{R}^2$:
	\[
	A = \{(0, y) \mid -1 \le y \le 1\} 
	\;\cup\;
	\{(x, \sin(1/x)) \mid 0 < x \le 1\}.
	\]
\end{exercise}

\solution{
	This is the topologist's sine curve mentioned the Intermezzo 2.
\begin{figure}[H]
	\centering
	\includegraphics[width=0.6\textwidth]{figures/topologistsinecurve.png}
	\label{fig:my_diagram}
\end{figure}
It is connected but not path connected and has two components.
\[
C_1 = \{(0, y) \mid -1 \le y \le 1\} 
\qquad
C_2 =\{(x, \sin(1/x)) \mid 0 < x \le 1\}.
\]
Each path component is contractible to a point. So 
$$
H_{0}= \mathbb{Z}\oplus\mathbb{Z}
$$
We see that there are no closed loops to make a cycles that are not boundaries, so 
$$
H_{1} = 0
$$
Since it is a 1 dimenstional surface embedded in $\mathbb{R}^{2}$ all higher homology groups are zero. This emphasizes the fact that $H_{0}$ is measuring path components, not components. We must be careful when these don't coincide.
}
\begin{exercise}
	Find explicitly a generator of $H_1(S^1)$.
\end{exercise}
\solution{
	We already know from class that
	\[
	H_1(S^1) \cong \mathbb{Z}.
	\]
	Consider the map
	\[
	\ell : \Delta^1 = [0,1] \longrightarrow S^1, 
	\qquad 
	\ell(t) = (\cos 2\pi t,\, \sin 2\pi t).
	\]
	It traces once around the circle in the counterclockwise direction.
The boundary of this simplex is given by
	\[
	\partial_1 \ell = \ell(1) - \ell(0) = (1,0) - (1,0) = 0.
	\]
	Hence $\ell \in Z_1(S^1) = \ker \partial_1$ is a $1$–cycle Now we show that $\ell$ is not a boundary. Every singular $2$–simplex $\sigma:\Delta^2\to S^1$ lies inside a small arc of the circle,
	and such an arc can be continuously shrunk to a point.
	Hence the loop $\partial_2\sigma$ can also be shrunk to a point;
	it represents $0$ in homology.
	Therefore $\operatorname{Im}\partial_2=0$, so $\ell$ is not a boundary.
\newline

	Since $H_1(S^1) \cong \mathbb{Z}$ and $[\ell]\neq 0$, it follows that
	$[\ell]$ must generate the entire group. 
	Geometrically, going once around the circle represents $1\in\mathbb{Z}$, 
	and going $k$ times around represents $k[\ell]$.
	\newline
	
	This concept appears widely in physics and biology:
	\begin{itemize}
		\item \textbf{DNA topology:} Closed DNA molecules behave like twisted loops. 
		The \emph{linking number} of two DNA strands measures how many times one winds around the other. 
		Enzymes called topoisomerases change this integer by cutting and reconnecting the strands.
		
		\item \textbf{Condensed matter physics:} 
		In systems with periodic boundary conditions (such as spins on a ring or a superfluid phase), 
		the winding number counts how many times the phase variable winds by $2\pi$ along the circle. 
		It classifies stable configurations such as vortices or persistent currents.
		
		\item \textbf{Quantum field theory:} 
		In field configurations with values in a circle or compact space (e.g.\ the 
		complex scalar field $\phi = e^{i\theta}$ in the $O(2)$ model), 
		the winding number is a topological charge conserved under continuous deformations. 
		Soliton and instanton numbers in higher dimensions generalize this idea.
	\end{itemize}
}
\lecture{12}{No Retract Theorem, Brouwer's Fixed Point Theorem, Local Homology and Topological Invariance of Dimension}
In the previous class we finished by using the excision theorem to compute the reduced homology groups of the spheres
\[
\tilde{H}_i(S^n) =
\begin{cases}
	\mathbb{Z}, & i = n,\\
	0, & i \neq n.
\end{cases}
\]
We now have the following corollary
\begin{corollary}
	The sphere $S^{n-1} = \partial D^n$ is not a retract of the disk $D^n$.
\end{corollary}

\begin{proof}
	Suppose, for the sake of contradiction, that there exists a retraction
	\[
	r : D^n \longrightarrow S^{n-1},
	\]
	that is, a continuous map satisfying
	\[
	r \circ i = \mathrm{Id}_{S^{n-1}},
	\]
	where $i : S^{n-1} \hookrightarrow D^n$ is the inclusion.
	
	Consider the induced maps in reduced homology:
	\[
	\widetilde{H}_{n-1}(S^{n-1})
	\xrightarrow{i_*}
	\widetilde{H}_{n-1}(D^n)
	\xrightarrow{r_*}
	\widetilde{H}_{n-1}(S^{n-1}).
	\]
	By functoriality of homology,
	\[
	(r \circ i)_* = r_* \circ i_* = (\mathrm{Id}_{S^{n-1}})_* = \mathrm{Id}_{\widetilde{H}_{n-1}(S^{n-1})}.
	\]
	
	However, we know from our computation of homology groups that
	\[
	\widetilde{H}_{n-1}(S^{n-1}) \cong \mathbb{Z},
	\qquad
	\widetilde{H}_{n-1}(D^n) = 0.
	\]
	Therefore, the inclusion map $i_*$ is the zero map. This implies
	\[
	r_* \circ i_* = 0.
	\]
	But the zero map cannot be the identity map on the group $\mathbb{Z}$. Hence, no such retraction $r$ can exist. So $S^{n-1}$ is not a retract of $D^n$.
\end{proof}
We can build on this proof to get Brouwer's fixed point theorem. Brouwer's fix pointe theorem shows us that if there isn't a fixed point then the retraction would exist.

\begin{theorem}[Brouwer Fixed Point Theorem]
	Let $D^n\subset\mathbb R^n$ be the closed unit disk
	$\{x\in\mathbb R^n:\|x\|\le 1\}$.
	Every continuous map $f:D^n\to D^n$ has a fixed point.
\end{theorem}


\begin{proof}
	We argue by contradiction.
	Assume that $f(x) \neq x$ for all $x \in D^n$.
	
	For each $x \in D^n$, consider the ray starting at $f(x)$ and passing
	through $x$. This ray intersects the boundary sphere
	$S^{n-1} = \partial D^n$ in exactly one point.
	Define a new map
	\[
	r : D^n \longrightarrow S^{n-1}
	\]
	by letting $r(x)$ be this intersection point.
	
\begin{figure}[H]
	\centering
	\includegraphics[width=0.4\textwidth]{figures/BrouwerFixedPoint.png}
\end{figure}
	
	Intuitively, $r(x)$ pushes each point $x$ outward from $f(x)$
	until it hits the boundary.
	
	Then $r$ is continuous, and for any $x$ already on the boundary,
	the ray from $f(x)$ to $x$ meets the boundary at $x$ itself.
	Hence
	\[
	r|_{S^{n-1}} = \mathrm{Id}_{S^{n-1}},
	\]
	so $r$ is a \emph{retraction}
	\[
	r : D^n \to S^{n-1}.
	\]
	
	But no such retraction can exists so $f$ must have a fixed point.
\end{proof}
The Brouwer Fixed Point Theorem has many concrete realizations in physics and related fields. 
In fluid dynamics, it implies that when stirring a cup of coffee, at least one particle of the fluid remains unmoved. 
In atmospheric or ocean flows (a related result on the sphere), there must always be at least one point with zero velocity. 
In thermodynamics and chemical kinetics, it guarantees the existence of an equilibrium state where system variables remain constant. 
In elasticity and continuum mechanics, any continuous deformation of a body that keeps the boundary fixed leaves at least one interior point unchanged. 
Finally, in game theory and economics (via the Nash equilibrium theorem), it ensures the existence of a stable equilibrium of strategies. Now we transition to local homologies. Throughout, we assume $X$ is a Hausdorff space.

\begin{definition}
	Let $z \in X$. The \emph{local homology groups of $X$ at $z$} are defined as
	\[
	H_p(X, X \setminus \{z\}).
	\]
\end{definition}

\begin{lemma}
	Let $X$ be a topological space and let $A \subseteq X$.
	Suppose $z \in A$ and $A$ is a neighborhood of $z$ 
	(that is, $A$ contains an open subset which contains $z$).
	Then
	\[
	H_p(X, X \setminus \{z\}) \;\cong\; H_p(A, A \setminus \{z\}).
	\]
\end{lemma}

\begin{proof}
	Let $U = X \setminus A$. 
	Then $z \notin \overline{U} = X \setminus \mathrm{Int}(A)$.
	Since $X$ is Hausdorff, we have 
	\[
	\overline{U} \subseteq X \setminus \{z\} 
	= \mathrm{Int}\big(X \setminus \{z\}\big).
	\]
	By the Excision Theorem,
	\[
	H_p(X, X \setminus \{z\})
	\;\cong\;
	H_p(X \setminus \overline{U}, (X \setminus \{z\}) \setminus \overline{U})
	\;\cong\;
	H_p(A, A \setminus \{z\}).
	\]
\end{proof}
To understand the meaning of the local homology groups 
\[
H_p(X, X \setminus \{z\}),
\]
let us consider a concrete example. Take $X = \mathbb{R}^2$ and $z = 0$. 
We want to understand the group
\[
H_p(\mathbb{R}^2, \mathbb{R}^2 \setminus \{0\}).
\]
The homology groups of the of the separate topological spaces are
\[
H_p(\mathbb{R}^2) =
\begin{cases}
	\mathbb{Z}, & p = 0,\\
	0, & \text{otherwise,}
\end{cases}
\qquad
H_p(\mathbb{R}^2 \setminus \{0\}) =
\begin{cases}
	\mathbb{Z}, & p = 0,\\
	\mathbb{Z}, & p = 1,\\
	0, & \text{otherwise.}
\end{cases}
\]
Removing the point introduces one $1$--dimensional hole around the origin.
The relative group $H_p(X, A)$ measures what is in $X$ but not already in $A$.
Equivalently, it is the homology of the quotient $X/A$ obtained by
\emph{collapsing} $A$ to a point. 
In our case, collapsing $\mathbb{R}^2 \setminus \{0\}$ to a point
turns the punctured plane into a $2$--sphere:
\[
(\mathbb{R}^2, \mathbb{R}^2 \setminus \{0\})
\;\simeq\;
(D^2, S^1).
\]
We know from standard results that
\[
H_p(D^2, S^1) =
\begin{cases}
	\mathbb{Z}, & p = 2,\\
	0, & \text{otherwise.}
\end{cases}
\]
Hence
\[
H_p(\mathbb{R}^2, \mathbb{R}^2 \setminus \{0\}) =
\begin{cases}
	\mathbb{Z}, & p = 2,\\
	0, & \text{otherwise.}
\end{cases}
\]
Removing the point $0$ from the plane creates a loop $S^1$ around the missing point.
Filling in that loop (i.e.\ restoring the point) corresponds to a $2$--dimensional
class, representing the ``local $2$--dimensional nature'' of the plane at that point.
Thus the local homology detects that $\mathbb{R}^2$ is locally a $2$--dimensional manifold. More generally,  $\mathbb{R}^n$ one has
\[
H_p(\mathbb{R}^n, \mathbb{R}^n \setminus \{0\}) =
\begin{cases}
	\mathbb{Z}, & p = n,\\
	0, & \text{otherwise.}
\end{cases}
\]
Therefore, the local homology at a point records the local topological dimension of the space. The group $H_p(X, X \setminus \{z\})$ measures how the topology of $X$ changes
when the point $z$ is removed.
It captures the ``shape'' of a small sphere around $z$, and thus reflects
the local structure of $X$ at that point.
In smooth manifolds, the local homology at each point is concentrated in the top degree,
confirming that every point has a neighborhood homeomorphic to $\mathbb{R}^n$.
\newline

The lecture derives the topological invariance of dimension using long exact sequences in homology for local homology groups and a simlar strategy to the proof of Brouwer's fixed point theorem, but I omit this proof. We can use these results to better understand geometry and manifolds with boundaries.
\begin{definition}
	Let $X$ be a Hausdorff space.
	\begin{enumerate}
		\item $X$ is an \emph{$n$-manifold} if each point of $X$ has an open neighborhood homeomorphic to an open subset of $\mathbb{R}^n$.
		\item $X$ is an \emph{$n$-manifold with boundary} if each point of $X$ has an open neighborhood homeomorphic to an open subset of 
		\[
		\mathbb{H}^n = \{ (x_1, \dots, x_n) \in \mathbb{R}^n \mid x_n \ge 0 \},
		\]
		the upper half-space of $\mathbb{R}^n$.
	\end{enumerate}
	The integer $n$ is called the \emph{dimension} of the manifold.
\end{definition}
\begin{remark}
	If $X$ is an $n$--manifold with boundary and $z \in X$, we have a chart
	\[
	f : U \subset X \longrightarrow V \subset \mathbb{H}^n
	\]
	which is a homeomorphism, with $U, V$ open subsets and $z \in U$.
	The map $f$ is called a \emph{chart} of $X$.
	
	We have two possibilities:
	\begin{enumerate}[i)]
		\item If $f(z) = (y_1, \dots, y_n)$ with $y_n > 0$, then
		\[
		H_p(X, X \setminus \{z\})
		\;\cong\;
		H_p(U, U \setminus \{z\})
		\;\cong\;
		H_p(V, V \setminus \{f(z)\})
		\;\cong\;
		H_p(\mathbb{R}^n, \mathbb{R}^n \setminus \{f(z)\})
		\;\cong\;
		\begin{cases}
			\mathbb{Z}, & p = n,\\
			0, & \text{otherwise.}
		\end{cases}
		\]
		
		\item If $f(z) = (y_1, \dots, y_n)$ with $y_n = 0$, then
		\[
		H_p(X, X \setminus \{z\})
		\;\cong\;
		H_p(\mathbb{H}^n, \mathbb{H}^n \setminus \{f(z)\})
		= 0
		\quad \forall p.
		\]
	\end{enumerate}
	
	Hence, if one chart of $z$ is of type (i) (resp. (ii)), all the charts of $z$
	are of the same type.
\end{remark}
This definition uses \emph{local homology} to distinguish interior and boundary points of a manifold:
interior points have local homology like $\mathbb{R}^n$, while boundary points have trivial local homology in degree~$n$.
It provides a purely \emph{topological characterization} of the boundary, showing that manifolds with boundary
are locally modeled either on $\mathbb{R}^n$ or on the half-space~$\mathbb{H}^n$.
\begin{definition}[Boundary of a manifold]
	Let $X$ be a manifold with boundary.  
	A point $x \in X$ is said to be \emph{contained in the boundary of $X$} if and only if there exists a chart
	\[
	\varphi : U \subseteq X \longrightarrow V \subseteq \mathbb{H}^n
	\]
	such that
	\[
	\varphi(x) = (y_1, \dots, y_n) \quad \text{with } y_n = 0.
	\]
\end{definition}
	By the previous results, this definition does not depend on the choice of the chart we consider.
The notion of boundary for a manifold is intrinsic and differs from the usual topological boundary.
Recall that for a subset $A \subseteq X$ of a topological space, the \emph{topological boundary} is defined as
\[
\partial_{\text{top}} A = \overline{A} \setminus \operatorname{int}(A),
\]
namely, the set of points whose every neighborhood intersects both $A$ and its complement.  The two notions generally do not coincide.  The topological boundary depends on how $M$ is embedded in an ambient space, while the manifold boundary depends only on the intrinsic differentiable structure of $M$.  
For instance, the open disk
\[
D^2 = \{ (x,y) \in \mathbb{R}^2 \mid x^2 + y^2 < 1 \}
\]
has empty manifold boundary $\partial_{\text{man}} D^2 = \varnothing$, even though its topological boundary in $\mathbb{R}^2$ is the circle $S^1$.  
On the other hand, the closed disk $\overline{D^2} = \{ (x,y) \mid x^2 + y^2 \le 1 \}$ satisfies $\partial_{\text{man}} \overline{D^2} = S^1$.  
Hence, the boundary of a manifold is not, in general, the same as the topological boundary.
\newline 

In conclusion, local homology provides a way to characterize both the dimension and the boundary structure of a manifold in purely topological terms.  
The dimension of a manifold is a topological invariant, meaning it is preserved under homeomorphisms.  
Moreover, if $f : M \to N$ is a homeomorphism between manifolds with boundary, then
\[
f(\partial M) = \partial N.
\]
Hence, the boundary of a manifold is also a topological invariant.  
These results emphasize that the notions of dimension and boundary depend only on the underlying topology of the manifold, not on any particular smooth structure or embedding.
\lecture{13}{Maeyer-Vitoris Sequence, Maps Between Spheres and Degree of the Map}
We will now build up to powerful tool for computing homology groups through sequences. However, we must first do some preliminary definitions.
\section*{Mayer--Vietoris Sequence}

\begin{definition}[Excisive couple]
	Let $X$ be a topological space and let $X_1, X_2 \subseteq X$ be such that
	\[
	X_1 \cup X_2 = X.
	\]
	Consider the inclusion map
	\[
	i : S_p(X_1) + S_p(X_2) \longrightarrow S_p(X),
	\]
	where $S_p(X)$ denotes the group of singular $p$-chains on $X$.
	(The groups $S_p(X_1)$ and $S_p(X_2)$ are regarded as subgroups of $S_p(X)$.) If, for each $p$, the inclusion map $i$ induces an isomorphism in homology, then the pair $\{X_1, X_2\}$ is called an \emph{excisive couple}.
\end{definition}

\begin{remark}
	We can equivalently write
	\[
	S_p(X_1) + S_p(X_2) = S_p(X)
	\]
	and denote $\mu = \{ X_1, X_2 \}$.
\end{remark}

\begin{theorem}[Sufficient condition for excision]
	If
	\[
	\operatorname{Int} X_1 \cup \operatorname{Int} X_2 = X,
	\]
	then the inclusion map
	\[
	i : S_p(X_1) + S_p(X_2) \hookrightarrow S_p(X)
	\]
	induces an isomorphism in homology by the Small Simplices Theorem. The Small Simplices Theorem states that given any open covering $\mathcal{U}$ of a topological space $X$, every singular simplex in $X$ is homologous to a sum of simplices whose images lie entirely within elements of $\mathcal{U}$.  
	In other words, the homology of $X$ can be computed using chains composed of ``sufficiently small'' simplices, each contained in a single open set of the covering.  
	This result ensures that when $\operatorname{Int} X_1 \cup \operatorname{Int} X_2 = X$, all homological information of $X$ is already captured by chains lying in $X_1$ or $X_2$, justifying the excision condition above.  
\end{theorem}
The requirement that the inclusion map
\[
i_* : H_p(S_p(X_1) + S_p(X_2)) \longrightarrow H_p(S_p(X))
\]
be an isomorphism means that, at the level of homology, the two subspaces $X_1$ and $X_2$ together capture all the essential topological information of $X$.  
Every cycle in $X$ can be represented by a combination of cycles lying in $X_1$ and $X_2$, and all relations among them already occur within their union.  
Thus, when the inclusion induces an isomorphism, passing from $X_1 \cup X_2$ to $X$ does not create or remove any homological feature — the decomposition $\{X_1, X_2\}$ is said to be \emph{excisive} because no ``piece'' of $X$ is missing from the union.
\begin{theorem}[Mayer--Vietoris Sequence]
	Let $X$ be a topological space and let $\{X_1, X_2\}$ be an excisive couple of $X$.
	Then there exists a long exact sequence in homology:
	\[
	\cdots 
	\longrightarrow H_p(X_1 \cap X_2)
	\xrightarrow{\ \Phi_p\ }
	H_p(X_1) \oplus H_p(X_2)
	\xrightarrow{\ \Psi_p\ }
	H_p(X)
	\longrightarrow H_{p-1}(X_1 \cap X_2)
	\longrightarrow \cdots
	\]
	where the connecting homomorphisms are given by
	\[
	\Phi_p(a) = \big(i_{1*}(a), -\,i_{2*}(a)\big), 
	\qquad
	\Psi_p(c, d) = k_{1*}(c) + k_{2*}(d),
	\]
	and $i_1, i_2, k_1, k_2$ denote the inclusion maps in the following commutative diagram:
	\[
	\begin{tikzcd}[row sep=large, column sep=large]
		X_1 \cap X_2 \arrow[r, "i_1"] \arrow[d, "i_2"'] 
		& X_1 \arrow[d, "k_1"] \\
		X_2 \arrow[r, "k_2"'] 
		& X
	\end{tikzcd}
	\]
	
	\noindent
	Moreover, if $X_1 \cap X_2 \neq \varnothing$, 
	then the Mayer--Vietoris sequence holds also for \emph{reduced homology}.
	\smallskip
	
	The sequence is also \emph{natural} with respect to continuous maps. Specifically, suppose $X$ and $Y$ are topological spaces such that
	\[
	X = X_1 \cup X_2,
	\qquad
	Y = Y_1 \cup Y_2,
	\]
	and let $f : X \to Y$ be a continuous map satisfying
	\[
	f(X_1) \subseteq Y_1
	\qquad \text{and} \qquad
	f(X_2) \subseteq Y_2.
	\]
	Then $f$ induces maps on the corresponding homology groups:
	\[
	f_* : H_p(X) \to H_p(Y), \quad
	(f|_{X_i})_* : H_p(X_i) \to H_p(Y_i), \quad
	(f|_{X_1 \cap X_2})_* : H_p(X_1 \cap X_2) \to H_p(Y_1 \cap Y_2),
	\]
	and these fit into the following commutative diagram:
	\[
	\begin{tikzcd}[column sep=small]
		\cdots \arrow[r] 
		& H_p(X_1 \cap X_2) \arrow[r, "\Phi_p"] \arrow[d, "(f|_{X_1 \cap X_2})_*"'] 
		& H_p(X_1) \oplus H_p(X_2) \arrow[r, "\Psi_p"] \arrow[d, "(f|_{X_1})_* \oplus (f|_{X_2})_*"'] 
		& H_p(X) \arrow[r] \arrow[d, "f_*"'] 
		& H_{p-1}(X_1 \cap X_2) \arrow[r] \arrow[d, "(f|_{X_1 \cap X_2})_*"'] 
		& \cdots \\
		\cdots \arrow[r] 
		& H_p(Y_1 \cap Y_2) \arrow[r, "\Phi_p"] 
		& H_p(Y_1) \oplus H_p(Y_2) \arrow[r, "\Psi_p"] 
		& H_p(Y) \arrow[r] 
		& H_{p-1}(Y_1 \cap Y_2) \arrow[r] 
		& \cdots
	\end{tikzcd}
	\]
\end{theorem}
The proof is quite boring, but it is very exciting to try it out on the spheres. Previously we worked very hard using the excision theorem to get the homology group of the spheres. Now we have a much better way.
\begin{exercise}
	Compute $\tilde{H}_i(S^m)$ by using the Mayer--Vietoris sequence.
\end{exercise}
We proceed inductively, namely that the homology groups for points are trivial for $p>0$ and $\mathbb{Z}$ for $p=0$. The direct sum of the homology groups of the disconnected points is $\mathbb{Z}\oplus\mathbb{Z}$ so  the reduced homology group $\tilde{H}_{0}(S^{0})\cong\mathbb{Z}$
\newline

Now consider the sphere $S^m$ decomposed into two hemispheres:
\[
S^m = Y_1 \cup Y_2,
\]
where $Y_1$ and $Y_2$ are the upper and lower closed hemispheres, whose intersection is the equatorial $(m-1)$-sphere.

	\begin{figure}[H]
	\centering
	\includegraphics[width=0.6\textwidth]{figures/sphereequator.png}
	\label{fig:my_diagram}
\end{figure}

We have
\[
\operatorname{Int} Y_1 \cup \operatorname{Int} Y_2 = S^m
\quad \Rightarrow \quad
\{Y_1, Y_2\} \text{ is an excisive couple.}
\]

Each $Y_i$ is homeomorphic to a closed $m$-disk $D^m$, hence contractible:
\[
Y_i \simeq D^m \quad \Rightarrow \quad Y_i \text{ are acyclic.}
\]

Moreover, the intersection $Y_1 \cap Y_2$ is homotopy equivalent to $S^{m-1}$:
\[
Y_1 \cap Y_2 \simeq S^{m-1}.
\]

	\begin{figure}[H]
	\centering
	\includegraphics[width=0.7\textwidth]{figures/equator.png}
	\label{fig:my_diagram}
\end{figure}

This is because
\[
Y_1 \cap Y_2 \approx S^{m-1} \times [-1,1],
\]
and $S^{m-1} \times \{0\}$ is a deformation retract of $S^{m-1} \times [-1,1]$,
via
\[
r(x, t) = (x, 0).
\]
We now apply the Mayer--Vietoris sequence for the decomposition
\[
S^m = Y_1 \cup Y_2, \qquad Y_1, Y_2 \text{ acyclic}, \qquad Y_1 \cap Y_2 \simeq S^{m-1}.
\]
The reduced Mayer--Vietoris sequence reads
\[
\cdots \longrightarrow 
\tilde{H}_p(Y_1 \cap Y_2)
\longrightarrow
\tilde{H}_p(Y_1) \oplus \tilde{H}_p(Y_2)
\longrightarrow
\tilde{H}_p(S^m)
\longrightarrow
\tilde{H}_{p-1}(Y_1 \cap Y_2)
\longrightarrow
\tilde{H}_{p-1}(Y_1) \oplus \tilde{H}_{p-1}(Y_2)
\longrightarrow \cdots
\]
Since $Y_1$ and $Y_2$ are contractible, all their reduced homology groups vanish:
\[
\tilde{H}_p(Y_1) = \tilde{H}_p(Y_2) = 0 \quad \forall p.
\]
Hence the sequence simplifies to
\[
0 \longrightarrow \tilde{H}_p(S^m)
\longrightarrow \tilde{H}_{p-1}(Y_1 \cap Y_2)
\longrightarrow 0,
\]
so that
\[
\tilde{H}_p(S^m) \cong \tilde{H}_{p-1}(Y_1 \cap Y_2)
\cong \tilde{H}_{p-1}(S^{m-1}).
\]
we conclude
\[
	\tilde{H}_i(S^m) =
	\begin{cases}
		\mathbb{Z}, & i = m,\\[4pt]
		0, & \text{otherwise.}
	\end{cases}
\]
Now we study some more topological properties of spheres.
\begin{definition}
	Let \( f : S^m \to S^m \) be a continuous map.  
	Consider the induced homomorphism in reduced homology:
	\[
	f_* : \tilde{H}_m(S^m) \longrightarrow \tilde{H}_m(S^m).
	\]
	Since \(\tilde{H}_m(S^m) \cong \mathbb{Z}\), we have
	\[
	f_* : \mathbb{Z} \longrightarrow \mathbb{Z}.
	\]

Therefore, there exists an integer \( d \in \mathbb{Z} \) such that
\[
f_*(\alpha) = d\,\alpha \quad \forall\, \alpha \in \mathbb{Z},
\qquad \text{equivalently, } f_*(1) = d.
\]
	The integer \( d \) is called the \emph{degree} of \( f \), denoted
	\[
	\deg(f) = d.
	\]
\end{definition}
\begin{remark}
	Since $\tilde{H}_m(S^m) \cong \mathbb{Z}$ is an additive group, any homomorphism
	$f_* : \mathbb{Z} \to \mathbb{Z}$ must satisfy
	\[
	f_*(a+b) = f_*(a) + f_*(b), \qquad \forall\, a,b \in \mathbb{Z}.
	\]
	The image of $1$ determines the whole map.  
	If $f_*(1) = d$, then by additivity
	\[
	f_*(n) = f_*(1+\cdots+1) = n f_*(1) = d n.
	\]
	Hence $f_*$ is necessarily ``multiplication by an integer $d$.''  For example, if $d=2$, then $f_*(3) = 6$.  
	A choice such as $f_*(3) = 5$ would force $f_*(1)=5/3$, which is not an integer
	and therefore not a valid group homomorphism $\mathbb{Z} \to \mathbb{Z}$.
	Thus the coefficient $d$ must always be an integer.
\end{remark}
We state the following properties of the degree without proof. They will be used in subsequent lessons to prove major topological results. Let $f, g : S^m \to S^m$ be continuous maps.  
\begin{enumerate}
\item \textbf{Normalization:}  
	\[
	\deg(\mathrm{id}_{S^m}) = 1.
	\]
	
	\item \textbf{Constant map:}  
	If $f$ is constant, then $\deg(f) = 0$.
	
	\item \textbf{Non-surjective maps:}  
	If $f$ is not surjective, then $\deg(f) = 0$.
	
	\item \textbf{Homotopy invariance:}  
	If $f \simeq g$ (i.e.\ $f$ and $g$ are homotopic), then
	\[
	\deg(f) = \deg(g).
	\]
	
	\item \textbf{Functoriality (composition rule):}  
	\[
	\deg(f \circ g) = \deg(f) \cdot \deg(g).
	\]
	
	\item \textbf{Homeomorphisms and homotopy equivalences:}  
	If $f$ is a homeomorphism (or a homotopy equivalence), then
	\[
	\deg(f) = \pm 1.
	\]
	Orientation-preserving maps have $\deg(f) = +1$,  
	orientation-reversing maps have $\deg(f) = -1$.
	
	\item \textbf{Degree of reflection:}  
	If $f$ is the reflection of $S^m$ in a hyperplane of $\mathbb{R}^{m+1}$, then
	\[
	\deg(f) = -1.
	\]
	(\emph{Geometric picture: the reflection reverses orientation.})
	
	\item \textbf{Degree and orientation change:}  
	If $\xi : S^m \to S^m$ is a homeomorphism, then for any $f$
	\[
	\deg(\xi \circ f \circ \xi^{-1}) = \deg(f).
	\]
	
	\item \textbf{Product of maps (optional, higher-level property):}  
	If $f : S^m \to S^m$ and $g : S^n \to S^n$, then
	\[
	\deg(f \times g) = \deg(f) \cdot \deg(g).
	\]
	
	\item \textbf{Hopf Theorem (1925):}  
	If $\deg(f) = \deg(g)$, then $f$ and $g$ are homotopic:
	\[
	\deg(f) = \deg(g) \quad \Longleftrightarrow \quad f \simeq g.
	\]
	
\end{enumerate}
\lecture{14}{Hairy Ball Theorem}
\begin{center}
	\textcolor{red}{all of lezione 22-11-21}
\end{center}






\newpage
\lecture{X}{All Remaining Problems}
\begin{exercise}
	Let $\Delta^3$ be the standard 3-simplex, and consider the topological space $X$ obtained by identifying the edge $[v_0, v_1]$ with the edge $[v_0, v_2]$ and $[v_1, v_3]$ with $[v_2, v_3]$.  
	Prove that $S^2$ is a deformation retract of $X$.  
	What identification of the edges of $\Delta^3$ gives a topological space having the torus $T^2$ as a deformation retract?  
	And what identification gives the projective plane $\mathbb{RP}^2$?
\end{exercise}

\begin{exercise}
	Compute the homology groups of $\mathbb{RP}^2$.
\end{exercise}

\begin{exercise}
	Let 
	\[
	0 \longrightarrow A \xrightarrow{f} B \xrightarrow{g} C \longrightarrow 0
	\]
	be a short exact sequence of abelian groups.
	
	\begin{enumerate}
		\item Prove that the following conditions are equivalent:
		\begin{enumerate}
			\item[(i)] There exists a homomorphism $g' : C \to B$ such that $g \circ g' = \mathrm{Id}_C$.
			\item[(ii)] There exists a homomorphism $f' : B \to A$ such that $f' \circ f = \mathrm{Id}_A$.
		\end{enumerate}
		If these conditions are satisfied, we say that the sequence \emph{splits}, or that it is a \emph{split exact sequence}.
		
		\item Prove that, if the sequence splits, then $B \cong A \oplus C$.
		
		\item Prove that, if $C$ is a free abelian group, then the sequence splits.
		
		\item Find an example of a short exact sequence that does \emph{not} split.
	\end{enumerate}
\end{exercise}

\begin{exercise}
	Let $X$ be the topological space obtained by identifying the four vertices of a square.
	Compute the homology groups of $X$.
\end{exercise}

\begin{exercise}
	Let $T^2$ be the $2$-dimensional torus. Assume that
	\[
	\widetilde{H}_1(T^2) \cong \mathbb{Z} \times \mathbb{Z}, 
	\qquad
	\widetilde{H}_2(T^2) \cong \mathbb{Z}, 
	\qquad
	\widetilde{H}_i(T^2) = 0 \text{ for } i \neq 1,2.
	\]
	(The homology groups of the torus can be computed by using the equivalence between simplicial and singular homology.)
	Compute the homology groups of $T^2 \setminus X_n$, where $X_n$ consists of $n$ distinct points of $T^2$.
\end{exercise}
\begin{exercise}
	Let $X$ be a non-empty topological space.  
	We define $\Sigma(X)$ as the topological space obtained from $X \times [0,1]$
	by collapsing $X \times \{0\}$ to a point and $X \times \{1\}$ to another point.  
	The topological space $\Sigma(X)$ is called the \emph{suspension} of $X$.
	
	\begin{enumerate}
		\item Prove that $\widetilde{H}_i(X) \cong \widetilde{H}_{i+1}(\Sigma(X))$.
		
		\item Prove that $\Sigma(S^n)$ is homeomorphic to $S^{n+1}$, and use this fact to compute
		the homology groups of the spheres.
		
		\item Let $f : X \to Y$ be a continuous function between topological spaces.
		Prove that $f$ induces a continuous function
		\[
		\Sigma f : \Sigma(X) \longrightarrow \Sigma(Y)
		\]
		defined by $\Sigma f(x,t) = (f(x), t)$.
		
		\item Prove that if $f : S^n \to S^n$ has degree $d$, then $\Sigma f : \Sigma(S^n) \to \Sigma(S^n)$
		has the same degree.
		
		\item Prove that for each $n \in \mathbb{Z}$ there exists a continuous function
		$f : S^1 \to S^1$ of degree $n$.
		
		\item Prove that for each $n, m \in \mathbb{Z}$ there exists a continuous function
		$f : S^m \to S^m$ of degree $n$.
	\end{enumerate}
\end{exercise}
\begin{exercise}
	Prove the \emph{Five Lemma}.  
	Are all the hypotheses in the statement presented during the lesson necessary?
\end{exercise}

\begin{exercise}
	\begin{enumerate}
		\item Let $P$ be a convex polyhedron with $v$ vertices, $l$ edges, and $f$ faces.  
		Prove that
		\[
		v - l + f = 2.
		\]
		
		\item A \emph{symmetry} of a polyhedron is an isometry of the Euclidean space
		that leaves the polyhedron invariant.  
		Two vertices of the polyhedron are said to be \emph{equivalent} if there exists a symmetry
		that maps one vertex to the other. Analogously, equivalence can be defined for edges and faces.
		
		A convex polyhedron is said to be \emph{regular} if all the vertices, all the edges, and all the faces are equivalent.  
		Prove that a regular convex polyhedron must be one of the following types:
		\[
		\text{tetrahedron, cube, octahedron, dodecahedron, or icosahedron.}
		\]
	\end{enumerate}
\end{exercise}

\begin{exercise}
	Using the results already obtained, compute the \emph{Euler characteristic} of the following spaces:
	\begin{itemize}
		\item the sphere $S^n$ of dimension $n$,
		\item the $2$-dimensional torus $T^2$,
		\item the torus with $n$ punctures (see Homework~6),
		\item the projective plane $\mathbb{RP}^2$.
	\end{itemize}
	What is the Euler characteristic of an \emph{acyclic space}?
\end{exercise}

\begin{exercise}
	Prove that the homology groups of the real projective space $\mathbb{RP}^n$ are given by
	\[
	H_i(\mathbb{RP}^n) =
	\begin{cases}
		\mathbb{Z}, & i = 0, \\[4pt]
		\mathbb{Z}, & i = n \text{ and } n \text{ odd}, \\[4pt]
		\mathbb{Z}_2, & 0 < i < n \text{ and } i \text{ odd}, \\[4pt]
		0, & \text{otherwise.}
	\end{cases}
	\]
\end{exercise}

\begin{exercise}
	Let $T^3$ be the topological space obtained by identifying the opposite faces of the cube 
	$[0,1]^3 \subset \mathbb{R}^3$ by translations (the space $T^3$ is called the \emph{3-dimensional torus}).  
	Prove that
	\[
	H_0(T^3) \cong H_3(T^3) \cong \mathbb{Z}
	\qquad \text{and} \qquad
	H_1(T^3) \cong H_2(T^3) \cong \mathbb{Z}^3.
	\]
\end{exercise}
\begin{exercise}
	\begin{enumerate}
		\item Prove that if the sequence
		\[
		A \xrightarrow{f} B \xrightarrow{g} C \longrightarrow 0
		\]
		is exact, then for each abelian group $G$ the following sequence is exact
		(right exactness of the tensor product):
		\[
		A \otimes G \xrightarrow{f \otimes \mathrm{Id}_G} 
		B \otimes G \xrightarrow{g \otimes \mathrm{Id}_G} 
		C \otimes G \longrightarrow 0.
		\]
		
		\item Prove that if the sequence
		\[
		0 \longrightarrow A \xrightarrow{f} B \xrightarrow{g} C \longrightarrow 0
		\]
		is exact, then for each free abelian group $H$ the following sequence is exact:
		\[
		0 \longrightarrow 
		A \otimes H \xrightarrow{f \otimes \mathrm{Id}_H} 
		B \otimes H \xrightarrow{g \otimes \mathrm{Id}_H} 
		C \otimes H \longrightarrow 0.
		\]
		
		\item Prove that the group $\mathbb{Z}_n \otimes \mathbb{Z}_m$ is isomorphic to
		$\mathbb{Z}_d$, where $d = \gcd(n,m)$.
	\end{enumerate}
\end{exercise}

\begin{exercise}
	Compute the homology groups of the following spaces:
	\[
	S^1 \times D^2 
	\qquad \text{and} \qquad 
	S^1 \times S^2.
	\]
\end{exercise}
\begin{exercise}
	Let $G$, $H$, and $(H_i)$ be abelian groups. Prove that:
	\begin{enumerate}
		\item $\operatorname{Tor}(H, G) = \operatorname{Tor}(G, H)$;
		
		\item $\operatorname{Tor}\!\left(\bigoplus_i H_i,\, G\right)
		= \bigoplus_i \operatorname{Tor}(H_i, G)$;
		
		\item If either $H$ or $G$ is torsion--free, then $\operatorname{Tor}(H, G) = 0$;
		
		\item $\operatorname{Tor}(H, G) = \operatorname{Tor}(T(H), G)$,  
		where $T(H)$ denotes the torsion subgroup of $H$;
		
		\item $\operatorname{Tor}(\mathbb{Z}_n, G) \cong \ker(m_n)$,  
		where $m_n : G \to G$ is defined by $m_n(g) = n g$;
		
		\item An exact sequence
		\[
		0 \longrightarrow H_1 \longrightarrow H_2 \longrightarrow H_3 \longrightarrow 0
		\]
		induces an exact sequence of the following type:
		\[
		0 \longrightarrow \operatorname{Tor}(G, H_1)
		\longrightarrow \operatorname{Tor}(G, H_2)
		\longrightarrow \operatorname{Tor}(G, H_3)
		\longrightarrow G \otimes H_1
		\longrightarrow G \otimes H_2
		\longrightarrow G \otimes H_3
		\longrightarrow 0;
		\]
		
		\item $\operatorname{Tor}(\mathbb{Z}_n, \mathbb{Z}_m) \cong \mathbb{Z}_d$,  
		where $d = \gcd(n, m)$.
	\end{enumerate}
	
	\medskip
	\noindent\textbf{Hint.}  
	Prove the properties in the following order:  
	(2), (5), (6), (3) for free abelian groups, (1), (3) for torsion–free abelian groups, (4), (7).
\end{exercise}
\begin{exercise}
	Compute the homology groups of the following spaces:
	\[
	S^1 \times D^2
	\qquad \text{and} \qquad
	S^1 \times S^2.
	\]
\end{exercise}

\begin{exercise}
	Let $X$ be a CW complex obtained by attaching two $2$--cells to the bouquet
	$S^1 \vee S^1$ according to the following attaching maps:
	\[
	a b a b a b a
	\quad \text{and} \quad
	a^4 b a^{-1}.
	\]
	Compute $H_i(X, G)$ for the following coefficient groups:
	\[
	G = \mathbb{Z},\ \mathbb{Z}_2,\ \mathbb{Z}_3,\ \mathbb{Z}_7,\ \mathbb{Q},\ \text{and}\ S^1,
	\]
	where $S^1$ denotes the multiplicative group of complex numbers of modulus one.
\end{exercise}

\begin{exercise}
	Compute the cohomology groups of $\mathbb{RP}^n$ with coefficients in $G$.  
	Present explicitly $H^i(\mathbb{RP}^n, G)$ for the same groups $G$ listed in the previous exercise.
\end{exercise}

\end{document}